% -----------------------------------------------------------------------------
% Review
% -----------------------------------------------------------------------------

\singlespacing
\begin{center}

{\Large \textbf{REVIEWS FOR:}}

\vspace{1cm}

{\Large \textbf{The visual environment and attention in decision making}}

\vspace{5mm}

% \textbf{authors...}

\vspace{1cm}
\end{center}


% -----
% Editor
% -----

\section{Editor}
\label{rev:editor}

\subsection{Overall evaluation}

\com[com-editor]{Thank you for submitting "The visual environment and attention in decision making" for review and consideration for publication in Psychological Bulletin. The editorial board has completed its review. . In addition to reading the manuscript myself, I was extremely fortunate to have received reviews from the same outstanding experts in the field who had reviewed your work before. I am grateful to them for their time and service to the field.\\
\\
The reviewers are now satisfied with your work. However, there are still a number of points that remain to be addressed to ensure that your paper meets the high standards required for publication in Psychological Bulletin. Accordingly, I am now inviting a revision that will address the following points:}

XXX


\com[com-editor-bias]{1- Given that you have now clarified that dependency among effect sizes occurred in your database, it is critical that you indicate clearly how your publication bias analysis addressed that dependency. Here are three papers that have studied the issue and provide direct guidance on appropriate methods:\\
\\
Fernández-Castilla, B., Declercq, L., Jamshidi, L., Beretvas, S. N., Onghena, P., \& Van den Noortgate, W.(2019). Detecting selection bias in meta-analyses with multipleoutcomes: A simulation study. The Journalof Experimental Education, 1–20.\\
\\
Rodgers, M. A., \& Pustejovsky, J. E. (In Press). Evaluating Meta-Analytic Methods to Detect SelectiveReporting in the Presence of Dependent Effect Sizes. Psychological Methods, forthcoming.https://doi.org/10.31222/osf.io/vqp8u\\
\\
Mathur, M. B., \& VanderWeele, T. J. (2020). Sensitivity analysis for publication bias in meta‐analyses. Journal of the Royal Statistical Society. Series C, Applied Statistics, 69(5), 1091.\\
\\
The second paper provides clear evidence that using publication bias tests that ignore dependency (as  done in the present analysis) can lead to grossly mistaken inferences. The third paper proposes a “sensitivity analysis” approach to examining publication bias in meta-analysis while accounting for effect size dependency. An R package that implements the approach can be found here: https://cran.r-project.org/web/packages/PublicationBias/index.html\\
\\
Although I do not expect you to implement all the approaches presented in these papers, I expect some adjustments of relevance in your paper. In fact, the trim and fill method is typically not appropriate for a multilevel model and should likely be removed.
}
    
XXX
    
    
\com[com-editor-independent]{2- Still related to this, given the dependency among effect sizes, it might be advisable to remove the word “independent” in the sentence stating “This resulted in 122 independent effect size estimates, out of which 50 were effects of visual factors and 72 were effects of cognitive factors…”, at the bottom of page 13.}

XXX


\com[com-editor-abstract]{3- The abstract seems quite brief. Perhaps you could provide more detail on the results (for example in terms of significant moderators)?}

XXX 


\com[com-editor-introend]{4- p. 12, The end of the introduction presents the results of the meta-analysis as if this was the discussion. This material does not belong there. Instead, the end of the introduction should summarize the main purpose of the meta-analysis, its importance to the field and theory, and make potential predictions on expected results (if they can be derived from the introduction).}

XXX


\com[com-editor-proofreading]{5- You should proofread carefully to reduce awkward phrasing and typos. Here are a few examples:
\begin{itemize}
    \item footnote 1, "There is also a growing number of studies using experimental manipulations to increase attention to random choice alternatives which seems to have a small positive effect on the chance of the alternative being chosen". That could be shortened. The structure “on the chance of the alternative being chosen” seems particularly awkward and should be revised. For example, would simply stating “response selection” cover this whole statement?
    \item p. 33, the heading "Publication bias exists, but its relatively small". Aside from the typo in “its”, it might be better just as “publication bias”.
    \item Method not methods.
    \item "Gray" not "Grey" (use American spelling).
    \item There is a typo in the title for Table 1: “parenthesis”, not “parentesis”.
\end{itemize}
}

xxx


\com[com-editor-variables]{6- Table 1 would suggest that you coded very few variables and I presume that it focuses on what you view as important moderators. I want to encourage you to be more explicit in presenting a list of all coded variables. A list of variables appears in the method but it is unclear whether other variables were considered (e.g., exploratory or descriptive factors). It seems good evidence synthesis practice to document the source of the papers that enter reviews/meta-analyses; in many such works, these provide a helpful context to interpret the results that then follow. For example, how about elaborating on the countries where the studies originated, what races/ethnicities and ages were represented in studies, and so on? Although these factors might not be plausibly connected to results, they could be used to address future directions and limitations in the context of the generalizability of the findings. This component would also make the discussion more sensitive to diversity factors. Alternatively, if studies are not describing their samples, then you could call on the literature to do so in future studies. It is also possible that these factors are not typically studied in this research area and that would have to be clarified as well. Regardless of whether you can provide an in-depth discussion of diversity variables, in the end, I would expect you to include a table that incorporates important features of the samples along with the moderators, as shown in the attached example (from a paper recently accepted for publication in Psychological Bulletin).}

XXX


\com[com-editor-limitations]{7- Related to the preceding point, the Discussion does not include a Limitations sub-section, which is typically expected in a review of this type. It could cover limitations of the included studies or of the methods used in evidence synthesis, but it requires discussion. On this point, the suggestion by Reviewer 2 that restrictions on language were imposed because the search terms were in English would count as a limitation that should be mentioned in the relevant section. For example, how much non-English literature might have been missed? Would these studies reach the same conclusions? Are important populations being omitted because of this particular method?}

XXX



% -----
% Reviewer 1
% -----

\section{Reviewer 1}
\label{rev:r1}

\com[com-r1-evaluation]{I was a reviewer on the original submission, and was largely positive about the ms. The authors have done an admirable job of considering my comments, and those of the other reviewers and editor in this revision. I am especially pleased by their attention and response to my concerns regarding the potential broad impact of their work. The paper is now in excellent shape, and suitable for publication.} 

XXX


% -----
% Reviewer 2
% -----

\section{Reviewer 2}
\label{rev:r2}

\com[com-r2-evaluation]{Dear Authors, thank you for submitting the revised version of your manuscript. I think that the authors provided very thorough responses to the reviewers' comments. All my concerns were allayed. I think the changes that the authors made improved the manuscript, which I now find suitable for publication.} 

XXX


% -----
% Reviewer 3
% -----

\section{Reviewer 3}
\label{rev:r3}

\com[com-r3-evaluation]{The reviewers have addressed all my concerns in a satisfactory manner. The paper has greatly improved and will now clearly make an important contribution to the literature.
I congratulate the authors to this very sophisticated work and recommend publication.} 

XXX


\clearpage


% -------------------------------------------------------
% Introduction
% -------------------------------------------------------

% part necessary to accommodate differences between review section and main text
\doublespacing
\setcounter{page}{1}
\setcounter{secnumdepth}{0}