% -------------------------------------------------------
% Discussion
% -------------------------------------------------------

\section{Discussion}

%%% 1. Brief reminder what the study is about
For the better part of our daily lives, we attend to and gather information using our eyes and consequently many of the decisions we make, small or large, depend on visual attention. In this article, we attempt to answer to what extent the visual environment guides our attention during decision making. To this end, we meta-analyze empirical studies on eye movements in decision making. We distinguish between visual environment factors such as salience, surface size, set size, and position, and compare them to cognitive factors such as preferential viewing, task instructions and choice-gaze effect. \chg{}{We identify 122 effect sizes from 69 articles} and perform a psychometric meta-analysis to control for methodological issues that arise when meta-analysing eye-movement studies.\\ 

% main findings - importance of visual factors 
% att odds with vast majority of current theories
Except for salience and left vs right position, the results show that visual factors have medium effect sizes. In comparison, effect sizes of the three cognitive factors are slightly larger, choice-gaze effect in particular. \chg{}{However, when controlling for publication bias, the visual factors are become larger than the cognitive factors.} In laboratory environments, it is possible, and often desirable, to control for visual factors, but in natural environments where no such control or counterbalancing takes place, all visual factors could influence eye movements simultaneously \citep{gidloef2017a, orquin2019a}. \chg{other-factors}{Furthermore, there are potentially other visual factors not covered in our study that influence eye movements. For instance, light and shade, texture, or occlusion by objects  \citep{geisler2008}, gestalt principles such as the laws of proximity, similarity, closure etc. \citep{wagemans2012}, or overall image properties such as feature or design complexity \citep{pieters2010a} or visual clutter \citep{rosenholtz2007a}. To the best of our knowledge none of these factors have been studied in the context of decision making.} Thus, visual factors might be major drivers of attention in real world decision making, well aligned with previous suggestions that 2/3 of variance in eye movements is due to visual factors \citep{vanderlans2008}. These findings are clearly at odds with most decision making models that assume equal attention to all stimuli \citep{tversky1979,payne1988, simon1956a}, but also with models that assume no role of cognitive factors in guiding attention in decision making \citep{busemeyer1992, krajbich2010a} or no role of visual factors in guiding attention \citep{callaway2019a, gloeckner2011a, gluth2018, gluth2020}.\\ 

% integrating visual and cognitive factors in models of attention and
% decision making
Our findings will hopefully reinvigorate the line of research integrating visual and cognitive factors in driving attention in decision making. Important first steps have been taken by \cite{chen2013}, \cite{navalpakkam2010}, and \cite{towal2013a}, who developed models integrating the role of salience in decision making. Their sequential sampling based models suggest that salience may influence the onset of drift or perhaps the amount of drift. This research left us with some important questions unanswered and new research should tackle these first. For example, we still do not know whether salience consistently biases attention in decision situations, or if the effect is limited to decisions under time pressure as in the before mentioned studies? If salience mainly influences attention immediately after stimulus onset \citep{theeuwes2010, orquin2015a}, the effect of salience on attention and choice may diminish as the decision time extends or it may have no bearing on the effect if salience influences the onset of drift as suggested by \cite{chen2013}. While there are still many unanswered questions about the mechanisms underlying the interactions between salience and decision processes, hardly any have been addressed concerning the other visual factors. Our findings are silent on the mechanisms and a pressing next step is to integrate multiple visual factors in decision making models to improve our understanding how exactly they jointly affect attention and possibly choices. A good starting point is to investigate visual factors with larger effect sizes identified in the present study -- surface size, center position, and set size -- alongside salience that has been studied previously. \chg{r427}{A promising modeling approach to account for visual and cognitive factors in decision making is to use similar model architecture for both attention and choice processes. Sequential sampling models are one option, as used in \cite{towal2013a}. Another option are interactive activation models that have been used for modelling both perceptual processes \citep{mcclelland1981} and choice processes, as in the parallel constraint satisfaction model \citep{gloeckner2011a}.} \chg{r426}{This model took an important step towards integrating the perceptual and choice domain and has, for instance, led to the discovery of the attraction search effect i.e. that decision makers are more likely to search attributes of already favored alternatives \citep{jekel2018}.} \\   

% what is a visual factor? limits of model-free definitions
% underscores the need for further modelling development
For the set size factor we observed the effect was moderated by alternative vs attribute, which reveals some limits of model-free classifications into visual and cognitive factors. We find a larger effect of set size by alternatives than set size by attributes, which implies that decision makers are more likely to ignore information when the set size increases in number of alternatives rather than in number of attributes. This finding suggests that, even though we have presented set size as a visual factor, it may influence the decision process as a cognitive factor, by moderating the search stopping point. Prior studies on multi-alternative decision making \cite{reutskaja2011, stuttgen2012, thomas2020} suggest that decision makers may rely on satisficing or a hybrid of satisficing for determining when to stop a search process. However, neither satisficing nor the proposed hybrid satisficing models can account for our findings on set size effects since these models assume that stopping is independent of the set size. This finding underscores the need for an integrative treatment of visual and cognitive factors in models of attention and decision making. This is the best way forward to improve our understanding of these findings and underlying mechanisms.\\

% external instructions and preferential viewing have the same 
% effects, further studies needed to examine whether attention
% process is really the same
Regarding cognitive factors, we decided to analyze studies on task instructions and preferential viewing separately since there is a clear qualitative difference between the two domains. In studies on task instructions, participants receive instructions concerning a specific decision goal, whereas, in preferential viewing studies, participants decide based on subjective preferences. The inspection of the effect sizes reveals that the main effect in the two types of studies are practically indistinguishable. This result suggests that it makes no difference to eye movements whether the relevance of information is defined according to an externally specified goal or according to subjective preferences. Breaking down both groups by alternatives and attribute moderators reveal further similarities. Although moderator analyses show a weak effect for preferential viewing and no effect for task instructions, in both cases there is a larger effect at the alternative level. An important caveat is that while effect sizes might be similar, the attention patterns behind them need not be. In other words, while both influence fixation count to a similar degree the order or timing of fixations could differ. Further research is necessary to determine whether preferential choice and choice according to external goals entail the same attention process as implied by, for instance, sequential sampling models \citep{forstmann2016}.\\ 

% choice-gaze effect, the biggest effect and still unresolved
Choice-gaze effect has the largest effect on eye movements in our study. The choice-gaze effect is similar for preferential and inferential studies, suggesting that the effect is not driven by preferential viewing. Even in tasks where participants are instructed to choose their least preferred alternative, they have more fixations to the chosen alternative. There are several theories predicting choice-gaze effect. One theory is that choice-gaze effect arises because of the gaze cascade phenomenon \citep{shimojo2003a}, but our findings suggest this cannot be the case since both preferential and inferential choices result in choice-gaze effect. Alternatively, choice-gaze effect could result from an evidence accumulation process as proposed in the attentional Drift Diffusion Model \citep{krajbich2010a}. The aDDM implies that the last fixation is often to the chosen alternative which could increase the fixation time or count for that alternative. However, effect size of the choice-gaze effect is substantial and most likely results from more than a single extra fixation to the chosen alternative. The aDDM is therefore not a good explanation for the choice-gaze effect phenomenon. Another possibility is that choice-gaze effect is the result of a process in which decision makers prioritize attention towards high-value alternatives as they learn about the values of the choice alternatives. There are several competing models that all imply a gradual orientation of attention towards high value alternatives \citep{callaway2019a, gloeckner2011a, manohar2013} and simulation studies may shed light on their ability to fully account for the choice-gaze effect phenomenon. A final possible explanation is that choice-gaze effect is the consequence of preparations for a motor response towards the chosen alternative \citep{hayhoe2014a}. This mechanism could furthermore contribute to choice-gaze effect along with other mechanisms such as the attention prioritizing process. The specific mechanism behind choice-gaze effect remains unclear; but considering how large the effect is, and the number of models that imply this effect, we believe that a better, and eventually full understanding of the effect will help advance decision research.\\ 

% Impact on broad range of disciplines
Our findings have implications for several scientific disciplines. Disciplines such as cognitive psychology, behavioral economics, and marketing are well represented in the set of included studies. For these disciplines, our findings provide a useful framework for developing successful behavioral interventions or marketing communication based on visual factors \citep{muenscher2016a, orquinwedel2020}. Our findings also point to the possibility of measuring individual preferences in real time through eye movements -- a technique that is becoming increasingly relevant as many everyday devices have built-in cameras that can serve as eye-trackers \citep{bulling2019a}. It is currently possible to perform low-resolution eye-tracking at home using a computer and web camera and preferential viewing could, for instance, serve as an implicit measure of preferences for a large sample of consumers. For vision science, our findings are particularly relevant being possibly the first meta-analysis to compare the effect of visual and cognitive factors on eye movements and may help refine gaze models of search \citep{vanderlans2008} and natural tasks \citep{hayhoe2005}. Other disciplines may want to take stock of these findings and to evaluate the generalizability of the findings to their respective discipline. Given the high degree of variance in methods and stimuli, we expect that our results generalize well to disciplines such as learning and education research, problem solving, or human-computer interaction. However, disciplines studying eye movements in natural environments, e.g., driving, aviation, or other natural tasks, should be cautious when applying our findings since the vast majority of the included effect sizes were from laboratory-based studies.\\ 

% \subsection{Methodological contributions}
Only a few meta-analyses have been published on eye movements and no guidelines exist on how to handle eye-tracking-specific issues in meta-analyses. To perform our analysis, we have developed procedures for how to handle issues related to multiple metrics and eye-tracker validity. The procedure for handling eye-tracker validity showed that eye-trackers with poorer accuracy, in general, lead to lower effect sizes. In our data, the difference in validity as indicated by the artefact multiplier ranged from .36 to .85 between the best and worst eye-trackers (see Table~\ref{tab:eyetracker_specifications}). This result is a substantial difference. Accounting for eye-tracker validity improved the precision of the synthesized effect sizes. This finding is an important methodological contribution which demonstrates the relevance of ensuring high-quality eye-tracker data. Eye movement related dependent variables come in multiple metrics such as fixation count, fixation likelihood, or dwell count. We showed that these metrics yield similar effect sizes and developed a method for converting effect sizes expressed in one metric into another. This method will allow future eye movement meta-analyses to overcome this important practical obstacle. From a methodological perspective, future research may further develop our framework for correcting for eye-tracker accuracy. The assumptions of our empirical method do not match the data perfectly and the method could be improved by taking into account the type of distributions of underlying dependent variables. Moreover, we know that several factors contribute to the validity of eye-trackers, e.g., data quality depends on the stimulus and the AOI size \citep{orquin2018a} and other artifacts such as sample population and recording location also matter \citep{nystroem2013a}. By extending our framework to include these other artifacts, it will be possible to make more precise estimates of effect sizes in meta-analysis and individual studies as well as more realistic power analyses.\\   

% \subsection{Limitations}
Some limitations of our findings have to be noted. All of the visual factors included a low number of studies which casts some doubt about the precision of the results. The low number of studies also means that the publication bias estimate is less reliable, thereby, adding to the uncertainty. This is unfortunate since recent findings suggest that meta-analytic results may considerably overestimate effect sizes compared to replication effect sizes, but that publication bias analysis largely reduces this difference \citep{kvarven2020}. An extenuating circumstance is that many of the included effect sizes were not central to or even hypothesized by the authors reporting them, which could imply that there was less selective reporting of these effects. One example is effect sizes for choice-gaze effect which many authors report as a by-product in descriptive statistics. Another challenge is that the studies included varied substantially e.g., high vs. low complexity stimuli or decision domain such as risky gambles vs. consumer choice. These differences may have introduced additional heterogeneity in the synthesized effect sizes, but at the same time, serve to increase the generalizability of the findings. \chg{r422c}{Finally, we wish to acknowledge that important process details may have been omitted from this analysis due to the selection of dependent variables. Specifically, two types of eye movement metrics are worth mentioning: fixation durations and scanpath metrics. Fixation durations can, for instance, reveal important aspects about decision processes such whether information is being processed in a more deliberate or intuitive manner \citep{horstmann2009}. However, not all studies report fixation durations and for our purpose the interpretation of fixation durations is complicated by the fact that it can reflect both visual and cognitive factors. Scanpath metrics such as the Search Index or Search Metric \citep{payne1976} are reported in even fewer studies, but can, for instance, be useful for distinguishing between compensatory and non-compensatory decision processes \citep{schoemann2019}.} \\ 

%%% 6. Finish with moving toward decision making in wild!
Our findings question several assumptions about how decision makers search for and gather information. The vast majority of existing theories and models assume, either implicitly or explicitly, that only cognitive factors matter. Most of the visual environment factors are ignored. While these models may work in a controlled laboratory environment, it is clear that they are not likely to generalize to more natural environments. \chg{call-to-action}{We hope our findings will inspire vision and decision scientists to collaborate on decision making models that integrate both visual and cognitive factors to improve our understanding of their interactions with the decision processes, and allow us to predict decision making in natural environments accurately. We also hope that decision researchers from the many disciplines included in this meta-analysis, such as psychology, neuroscience, economics, marketing, management, and political science, will become aware of the role of visual factors in their own context and examine visual factors with representative designs rather than experimentally eliminating them.}
