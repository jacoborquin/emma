% -----------------------------------------------------------
% Method
% -----------------------------------------------------------

\section{Methods}

\subsection{Literature search}

Web of Science was searched using the following terms: eye track* OR eye move* OR eye fix* AND decision making OR choice. Grey literature, such as reports and unpublished work, was identified in the first 2,000 hits on Google Scholar. No restrictions on publication date or language were imposed. Additional literature was identified by searching the reference lists of the identified papers and through contact with the authors. Calls for unpublished studies were distributed to the relevant research communities via email lists during February 2018 at the following lists; European Association for Decision Making (EADM), Society for Judgment and Decision Making (SJDM), and European Group of Process Tracing Studies (EGPROC). The search resulted in 291 studies screened for eligibility. The last search was done on March 1st, 2018.


\subsection{Inclusion criteria}

We included studies in which participants made decisions or judgments between discrete alternatives while their eye movements were recorded using eye-tracking technology. We did not include studies related to perceptual judgments, such as categorizing or discriminating visual stimuli or studies on problem solving. We excluded studies where participants were selected based on clinical diagnosis or specific socio-demographic traits e.g., visual disorders, age-related visual diseases, age restrictions such as adolescents or infants. Studies using fixed exposure time or time pressure manipulations were excluded since these manipulations can influence eye movement processes \citep{orquin2018a} and lead to substantially different results \citep{simola2019a}. Included studies used either fixation likelihood (area of interest (AOI) looked at or not), fixation count (number of fixations to AOI), total fixation duration (sum of durations of all fixations to an AOI), or dwell count (number of dwells to an AOI). Eventually, 58 articles met all inclusion criteria and were included in the meta-analysis (Figure~\ref{fig:flow_diagram}).


\subsection{Data extraction and coding procedure}

The included studies were coded with regards to their (1) effect size, (2) sample size, (3) research domain, (4) eye-tracker model, (5) dependent variable, and (6) independent variable. All studies were initially coded by the ESL and later by JLO. Any disagreement was resolved by discussion. Agreement for categorical variables was assessed using Cohen's kappa and for continuous variables using intraclass correlation coefficient \citep{shrout1979a}. Overall, there was a high level of agreement: effect size, $\textrm{ICC} = 0.923$, sample size, $\textrm{ICC} = 0.996$, research domain, $\textrm{ICC} = 0.731$, eye tracker model, $\textrm{ICC} = 1$, dependent variable, $\kappa = 0.923$, independent variable, $\kappa = 0.934$.

Coding of effect sizes is described in detail below and sample size was coded as the total number of participants in a study. The research domain was coded as preferential consumer choice, inferential consumer choice, preferential non consumer choice, inferential non consumer choice, and risky gambles. The research domain was later recoded for the analysis of choice bias in the following way: inferential consumer choice and inferential non consumer choice were recoded as inferential choice while the other three domains were coded as preferential choice. We coded the eye-tracker model as the specific name of the eye-tracking equipment used in the study, e.g. Tobii T2150 or Tobii T60, since different models from the same producer vary in measurement accuracy and precision. Information on each eye-tracker model's accuracy and precision was identified through the equipment producers' websites. We coded the dependent variable as the specific eye-tracking metric in which an effect size was reported. We coded the independent variable as visual or cognitive factors, with visual factors divided into five dimensions -- salience, surface size, left vs right position, central position, and set size -- and cognitive factors divided into three dimensions -- task instructions, preferential viewing, and choice bias. We outline these categories in detail below. 

\paragraph{Salience.} We coded studies as salience if they operationalized one or more of the known dimensions of salience such as color, edge density, contrast, or motion \citep{itti2000}. Some studies failed to indicate the direction of the salience manipulation, i.e. high vs. low levels of salience. In such cases, we contacted the original author and asked for clarification.

\paragraph{Surface Size.} We coded studies that manipulated the relative surface size of alternatives or attribute, e.g., small vs. large alternatives or attributes \citep{lohse1997a}. Some studies manipulated the number of product facings, i.e., the number of the same product on a supermarket shelf \citep{chandon2009a}. We coded such manipulation as a surface size manipulation. 

\paragraph{Left vs right and center position.} We coded studies that manipulated the left vs right position of alternatives or attributes in horizontal arrays as left vs right position \citep{kreplin2014a}. We coded studies that manipulated the centrality of alternative or attribute position in one or two-dimensional arrays as center position \citep[experiment 1A \& 1B in][]{atalay2012a,meissner2016a}.

\paragraph{Set size.} We coded studies as set size if they manipulated the number of alternatives or attributes in a given choice task, e.g., studying the effect of a two- vs. three-alternative choice task \citep{hong2016a}. We also coded whether the set size was manipulated at the level of the alternative or the attribute. 

\paragraph{Task instruction.} We coded studies on task instruction if they presented participants with identical stimuli under different task instructions, e.g., testing the effect of a preferential  vs. inferential choice on eye movements \citep{orquin2019a}. We also coded whether the unit of analysis was at the level of the alternative or the attribute, i.e. whether AOI's contained alternatives or attributes. 

\paragraph{Preferential viewing.} We coded studies on preferential viewing if they measured the effect of preferences on eye movements. In these studies, preference was either measured in an independent task (e.g. Becker-DeGroot-Marschak auction) or revealed through a choice in the choice task (i.e. chosen vs non-chosen alternative). We also coded whether the unit of analysis was at the level of the alternative, e.g. when participants prefer one alternative over another because it is cheaper or has a better flavor \citep{gidloef2017a}, or at the level of attributes, e.g. when price is more important than flavor \citep{meissner2016a}. 

\paragraph{Choice bias.} We coded studies as choice bias if they reported the difference in eye movements between the chosen alternative and all other (not chosen) alternatives. Studies that operationalized choice bias in specific time windows, e.g., the first 500 msec after stimulus onset or last 500 msec prior to choice \citep{shimojo2003a} were excluded. Based on the research domain we coded choice bias in two subfactors: preferential tasks where participants performed a preferential choice task, that is where participants were instructed to choose in accordance with their preferences \citep{schotter2010a} and inferential tasks where participants were instructed to choose in accordance with a predetermined goal, such as choosing the healthiest alternative \citep{schotter2012a}.


\subsection{Construct validity of the dependent variable}

A possible concern in meta-analyses of eye movements is that the included studies use different eye-trackers, since data quality varies considerably across different eye-tracking equipment. Precision, which is the reliability of an eye-tracker, can vary as much as from $.005\degree$ root mean square in the best to $.5\degree$ in the poorest remote eye-trackers \citep{holmqvist2015a}. Accuracy, which is the validity of an eye-tracker, vary from around $.4\degree$ to around $2\degree$ \citep{holmqvist2015a}. With an accuracy of $2\degree$, the measured fixation, will on average fall as far as $2\degree$ away from the true fixation point. Simulations have shown that both accuracy and precision influence the capture rate, i.e., the percentage of eye movements correctly recorded within the boundaries of stimuli, which determines the degree of false positive and false negative observations \citep{orquin2019a}. The level of false positive vs. negative fixations has been shown to influence effect sizes \citep{orquin2016a}. These differences in measurement validity across eye-trackers may therefore introduce a bias in the meta-analysis of eye movements, since studies with lower accuracy and precision have lower validity, which, on average, attenuate effect sizes (Hunter \& Schmidt, 2004). To inspect whether the precision and validity of eye-trackers attenuate effect sizes, and potentially correct for this, we ran a regression analysis on all included effect sizes with the absolute observed effect size correlation as the dependent variable and reported precision and accuracy of the eye-tracking equipment as the independent variables. We fitted different models using a step-up approach \citep{ryoo2011model} based on Bayesian information criterion \citep{Schwarz1978}, including models with a fixed effect for the independent variable type (salience, surface size etc.). The final model included the main effect of accuracy and a random intercept grouped by study. The second-best model also included a fixed effect for independent variable type, and the estimates of the two models were comparable. The accuracy and precision of eye-trackers are highly correlated ($r = .63$), and presumably for this reason model fit did not improve when including precision. Despite analyzing across different study factors and other sources of noise, the results suggest that studies using eye-trackers with lower levels of accuracy, on average, yield lower effect sizes as predicted by the psychometric meta-analysis methods, ($\beta_0=0.429$, $\SE=0.128$, $t=3.349$, $p=0.001$, $\beta_{\textrm{accuracy}} =-0.192$, $\SE=0.13$, $t=-1.478$, $p=0.149$; Figure~\ref{fig:ET_accuracy_effectsize}). Having demonstrated that the accuracy of eye-trackers attenuates effect sizes, the next step is to correct for this phenomenon. Psychometric meta-analysis offers a method for correcting the attenuating effects of artifacts, such as the lack of validity or reliability \citep{hunter2004a}. The correction involves an artifact multiplier, $a_a$, which is a measure of the expected attenuation of the true effect size $\rho$ caused by the artifacts in study $i$. The observed study effect size $\rho_0$ is a function of the true effect size and the artifact multiplier, $\rho_0 = a_a \rho$. In the case of measurement validity, the artifact multiplier is the square root of the validity of the measurement, $a_a = \sqrt{r_{yy}}$. From this calculation, it follows that the artifact multiplier, and, hence the validity of the measurement, can be obtained as $a_a = \rho_0 / \rho$ \citep{hunter2004a}. From our model, we have estimated the observed attenuated effect size, $\rho_0$, of study $i$ as $\beta_0 + \beta_1 \textrm{accuracy}$. Given perfect accuracy, i.e. accuracy takes the value zero, the expected effect size of study $i$ is equal to the intercept, $\beta_0$, which corresponds to the expected unattenuated effect size, $\rho$. From this it follows that the artifact multiplier, $a_a$, can be computed as the ratio of the attenuated effect size proportional to the unattenuated effect size:
%
\begin{equation}
\label{eq:artifact_multiplier}
a_a = \frac{\beta_0 + \beta_1 \textrm{accuracy}}{\beta_0}
\end{equation}

For example, if a study uses an eye-tracker with an accuracy of $.50$, this yields an artifact multiplier equal to $(.569 - .382*.50)/.569 = .664$, meaning that studies with this level of accuracy will, on average, experience effect sizes that are $66.4\%$ of the true population effect size $\rho$. To compute the true average effect, $\rho$, we follow the psychometric meta-analysis method proposed by \cite{hunter2004a}. We first compute the unattenuated effect size correlation for each study, $r_i^u$, by dividing the Fisher transformed attenuated effect size with the artifact multiplier that corresponds to the level of the eye-tracker accuracy and then applying the inverse Fisher transformation, $r_i^u = \tanh(\arctanh(r_i)/a_a)$. An issue with correlation coefficients is that effect of multiplication depends on the value of the coefficient, particularly near the boundaries (-1 and 1), Fisher transformation alleviates this issue. Then, we weight each study by its sample size and its level of validity, so that studies using low accuracy eye-trackers are corrected upwards, in terms of their effect sizes and variance (Equation~\ref{eq:psychometric_rho}). A full list of eye-trackers and their accuracy and precision can be found in Table~\ref{tab:eyetracker_specifications} in Appendix~\ref{appendix}.


\subsection{Multiple metrics}

Another possible concern in meta-analyses of eye movements is that studies often rely on different eye movement metrics as their dependent variable. However, to perform a meta-analysis, we need to compare studies across a common dependent variable. The many different eye movement metrics stem from different research designs and research questions and, perhaps, also a lack of consensus about when and why to use which metrics. Many studies on visual factors report fixation likelihood while studies on cognitive factors often report fixation or dwell count. We focus on fixation count as it is easier to interpret than both the total fixation duration and the dwell count. The total fixation duration can, for instance, be difficult to interpret when there is a correlation between the fixation duration and the fixation count \citep{orquin2018a}. The dwell count, defined as continuous fixations within same AOI without switching elsewhere, is similarly difficult to interpret if there is a correlation between the number of or the duration of fixations per dwell and the probability of a dwell, in which case the number of dwells is not indicative of the number of fixations. In order to inspect whether it would be meaningful to average effect sizes across different eye-tracking metrics, we reviewed the identified articles for studies that reported effect sizes in multiple metrics. We identified in total $43$ studies reporting fixation likelihood along with one additional metric and $48$ studies reporting fixation count along with one additional metric. To investigate the strength of the relationship between the metrics, we inspected the linearity of the relationship between fixation likelihood and fixation count against other metrics by plotting all observations (Figure~\ref{fig:metric_correction}). Since the four eye movement metrics are highly correlated, we assume that the metrics are related to the same underlying construct.\\ 
While effect sizes expressed in different metrics are highly correlated, we should expect some differences between them. One mechanism that could lead to differences in effect size estimates between fixation likelihood and the remaining metrics is artificial dichotomization since fixation count, dwell count and total fixation duration are treated as a binary outcome (fixated or not fixated) to produce fixation likelihood. Artificial dichotomization of a naturally continuous variable attenuates correlations with other variables \citep{hunter2004a}. We should, therefore, expect effect sizes expressed in fixation likelihood to be somewhat smaller. Correcting for artificial dichotomization requires knowledge about the true distributional split. Since none of the included studies provide information about the true distributional split of the dichotomization and since we do not have access to all data sets, we are unable to compute the artifact multiplier as proposed by \cite{hunter2004a}. Furthermore, since the eye-tracking metrics are distributed according to either zero inflated normal distribution (total fixation duration) or Poisson distribution (fixation and dwell count), no such adjustments for dichotomization currently exist. Instead, we propose an empirically derived correction factor, $a_m$, to convert effect sizes expressed in one metric to another. We propose to estimate the correction factor based on our sample of studies reporting multiple metrics, by taking the ratio of the sample size weighted means expressed in the two metrics of interest:
%
\begin{equation}
\label{eq:metrics_correction}
a_m = \frac{\arctanh \left( \frac{\sum M_i^1 N_i}{\sum N_i} \right)}{\arctanh \left( \frac{\sum M_i^2 N_i}{\sum N_i} \right)}
\end{equation}
%
where $\arctanh \left( \frac{\sum M_i N_i}{\sum N_i} \right)$ is the Fisher transformed average effect size for metric $M^1$ and $M^2$, respectively weighted by sample sizes, $N$ in study $i$. The ratio is computed on the Fisher transformed effect sizes in order to meaningfully compare ratios across the whole range of correlations. For similar reasons, the correction factor is applied to Fisher transformed effect sizes which are then transformed back with the inverse Fisher transformation: $\tanh(\arctanh(r_i)*a_m)$. The method takes advantage of the fact that effect sizes from the same study expressed in different metrics control for all factors that could influence the ratio.\\    
As expected, we find that effect sizes reported in fixation likelihood are on average smaller than those reported in metrics that are not artificially dichotomized, i.e. fixation count, dwell count, and total fixation duration. An effect size estimate expressed in fixation likelihood is, for instance, $97.2\%$ of the effect expressed in fixation count. Table~\ref{tab:metric_correction} shows an overview of the correction factor $a_m$, that needs to be applied to convert different metrics to either fixation likelihood or fixation count. We expressed all metrics in fixation counts by applying the correction factor to each individual study effect size, but not to the study variance. When a study effect size is already reported fixation count, $a_m$ takes the value $1$.  


\subsection{Publication bias}

The relationship between effect size and its standard error in each group was inspected visually using funnel plots (Figure~\ref{fig:funnel_plots} and Figure~\ref{fig:funnel_plots_altatt}). The trim and fill method was used to take into account potential impact of any publication bias \citep{duval2000trim}. This method imputes studies to achieve a symmetric distribution of effect sizes and then computes the synthesized effect size including the imputed studies.


\subsection{Statistical analyses}

\subsubsection{Computation of effect sizes}

Effect size information was transformed into a common effect size, the Pearson’s correlation coefficient r. When multiple sources for computation of effect sizes were available, priority was given in decreasing order to other effect size measures, means and standard deviations, test statistics, beta coefficients, or p values. For studies reporting effect sizes as correlations, no further computations were performed. If a study reported p values as a threshold value, e.g., $p < .05$, we used a conservative p value equal to .05. When studies reported effect sizes for multiple AOI's, we computed the average effect size across AOI's \citep[for a similar approach, see][]{chita2016attention}. Effect sizes were extracted from the available dependent variables. Analyses were performed in R programming language with the help of several additional libraries \citep{R2020,del2012a,datatable,ggplot2,metafor}.


\subsubsection{Weighting of effect sizes, tests of heterogeneity}

The effect sizes were analyzed with a psychometric meta-analysis following the approach in \cite{hunter2004a}. Individual effect sizes were first corrected using the metric correction factor, $a_m$, to yield a common dependent variable. All studies were corrected to fixation count. The psychometric meta-analysis computes the true average effect size $\rho$ based on the unattenuated correlation coefficients, $r_i^u$, weighted by sample size $n_i$, and corrected for validity by the artifact multiplier, $a_a$: 
%
\begin{equation}
\label{eq:psychometric_rho}
\rho = \frac{\sum_{i=1}^k n_i a_a^2 r_i^u}{\sum_{i=1}^k n_i a_a^2}
\end{equation}

To inspect the degree of heterogeneity in the meta-analysis, we computed the $I^2$ statistic. The $I^2$ is the proportion of variance in the observed (attenuated) effect estimates explained by artifacts and sampling error \citep{borenstein2011introduction}: 
%
\begin{equation}
\label{eq:i2_statistic}
I^2 = \frac{(T^u)^2}{(S^u)^2}
\end{equation}
%
where $(S^u)^2$ is the weighted variance of the unattenuated effect size $\rho$
%
\begin{equation}
\label{eq:Su2_var}
(S^u)^2 = \frac{\sum_{i=1}^k n_i a_a^2 (\rho_i - \hat{\rho})^2}{\sum_{i=1}^k n_i a_a^2}
\end{equation}
%
and $(T^u)^2$ is the between-studies variance component of the unattenuated effect size $\rho$
%
\begin{equation}
\label{eq:Tu2_var}
(T^u)^2 = (S^u)^2 \frac{\sum_{i=1}^k n_i a_a^2 v_i}{\sum_{i=1}^k n_i a_a^2}
\end{equation}
%
where $v_i$ is the variance of study $i$ computed as $(1 - \hat{r}^2)^2 / (n_i - 1)$ and $\hat{r}$ is the sample size weighted average effect size.
