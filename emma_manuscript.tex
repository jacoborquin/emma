%-*- program: pdflatex -*-
%-*- program: bibtex -*-
%-*- program: pdflatex -*-
%-*- program: pdflatex -*-


% --------------------------------------------------------------
% Preamble
% --------------------------------------------------------------

\documentclass[english,natbib,man,floatsintext,mask]{apa6}
% \usepackage{arxiv}
\usepackage[english]{babel}
\usepackage[utf8]{inputenc} % allow utf-8 input
\usepackage[T1]{fontenc}    % use 8-bit T1 fonts
\usepackage{url}            % simple URL typesetting
\usepackage{hyperref}       % hyperlinks
\usepackage{xcolor}
\hypersetup{
    colorlinks,
    linkcolor={red!50!black},
    citecolor={blue!50!black},
    urlcolor={blue!80!black}
}
\usepackage{booktabs}       % professional-quality tables
\usepackage{tabularx}
\usepackage{amsfonts}       % blackboard math symbols
\usepackage{nicefrac}       % compact symbols for 1/2, etc.
\usepackage{microtype}      % microtypography
\usepackage{gensymb}        % degree, angle symbols

% for math equations and symbols
\usepackage{amsmath} 
\usepackage{amssymb} 
\newcommand{\E}{\mathbb{E}}
\newcommand{\SD}{\mathit{SD}}
\newcommand{\SE}{\mathit{SE}}
\newcommand{\BF}{\mathit{BF}}
\DeclareMathOperator\arctanh{arctanh}

% This prevents placing floats before a section.
\usepackage{placeins}

% bibliography
% \usepackage{natbib}

% allows for floats when doing jou or doc style
\usepackage{graphicx}      
\graphicspath{{./figs/}}  
%\usepackage{float}

% for long tables
\usepackage{longtable}

% linenumbers
\usepackage[mathlines]{lineno}

% in APA man mode captions are too large
% making sure they are reasonable
\usepackage{setspace}
% \usepackage[font=singlespacing]{caption}
% \captionsetup{font=singlespacing}

%%% some support for commenting

\usepackage{color}
\definecolor{Blue}{RGB}{0,0,255}
\definecolor{Red}{RGB}{255,0,0}
\newcommand{\jo}[1]{\textcolor{Red}{[Jacob: #1]}}  
\newcommand{\hs}[1]{\textcolor{Blue}{[Hrvoje: #1]}} 
% \newcommand{\jo}[1]{}  
% \newcommand{\hs}[1]{} 

%%% commands for help with reviews

% marking the reviewer comments
\newcommand\eatpunct[1]{}
\newcommand{\com}[2][]{\vspace{5mm}\paragraph[ ]{ \eatpunct}\label{#1}\emph{#2}\vspace{5mm}}

% \par\refstepcounter{paragraph}\paragraphmark{#1}

% marking the changes in the main text
% for hyperlinks, see: https://tex.stackexchange.com/questions/129541/hyperref-and-lineno-always-jumping-to-first-page
\newcommand{\llabel}[1]{\hypertarget{llineno:#1}{\linelabel{#1}}}
\renewcommand{\lineref}[1]{\hyperlink{llineno:#1}{\ref*{#1}}}
\newcommand{\chg}[2]{\llabel{#1}\textcolor{Red}{ #2}} 
\newcommand{\quotetext}[1]{\vspace{2mm}\noindent \hangindent=-2cm \hangafter=0 ``#1''\vspace{2mm}}


% ---------------------------------------------------------
% Title, authors
% ---------------------------------------------------------

\title{The visual environment and attention in decision making}

\shorttitle{The visual environment and attention in decision making}

\threeauthors{Jacob L. Orquin*}{Erik S. Lahm}{Hrvoje Stojić}
\threeaffiliations{Aarhus University and Reykjavik University}{Aarhus University}{University College London}

%This description should be included within the manuscript on the abstract/keywords page.

\authornote{Jacob L. Orquin and Erik S. Lahm, Department of Management/MAPP, Aarhus University, Fuglesangs alle 4, 8210 Aarhus V - Denmark; Hrvoje Stojić, Max Planck UCL Centre for Computational Psychiatry and Ageing Research, University College London, 10-12 Russell Square, London, WC1B 5EH, United Kingdom. The authors thank Martin Meissner, Tobias Otterbring, Sonja Perković, and Valdimar Sigurdsson. This research was supported by the Independent Research Fund Denmark, grant number: 8046-00014A and the Lundbeckfonden, grant number R281-2018-27. \\ 
Data Availability Statement. The data and code reported in the manuscript is made available at Open Science Framework at \url{https://osf.io/buk7p/} \citep{orquin2020osfa}. \\
*Correspondence concerning this article should be addressed to Jacob L. Orquin, Department of Management/MAPP, Aarhus University, Fuglesangs alle 4, 8210 Aarhus V - Denmark. E-mail: jalo@mgmt.au.dk.}

%\rightheader{Orquin, Lahm, Stojic}
%\leftheader{The visual environment and attention in decision making}


% ---------------------------------------------------------
% Abstract
% ---------------------------------------------------------

\abstract{% limit: 250 words
Visual attention is a fundamental aspect of most everyday decisions, and governments and companies spend vast resources competing for the attention of decision makers. In natural environments, choice options differ on a variety of visual factors, such as salience, position, or surface size. However, most decision theories ignore such visual factors, focusing on cognitive factors such as preferences as determinants of attention. \chg{}{To provide a systematic review of how the visual environment guides attention we meta-analyze 122 effect sizes on eye movements in decision making. The psychometric meta-analysis and $Top10$ sensitivity analysis show that visual factors play a similar or larger role than cognitive factors in determining attention. The visual factors that most influence attention are positioning information centrally, $\rho = .43$ $(Top10 = .67)$, increasing the surface size, $\rho = .35$ $(Top10 = .43)$, reducing the set size of competing information elements, $\rho = .24$ $(Top10 = .24)$, and increasing visual salience, $\rho = .13$ $(Top10 = .24)$. Cognitive factors include attending more to preferred choice options and attributes, $\rho = .36$ $(Top10 = .31)$, effects of task instructions on attention, $\rho = .35$ $(Top10 = .21)$, and attending more to the ultimately chosen option, $\rho = .59$ $(Top10 = .26)$.} Understanding real-world decision making will require integration of visual and cognitive factors in future theories of attention and decision making.}
% limit: 5 keywords
\keywords{eye movements, decision making, attention, meta-analysis, visual environment, cognitive factors} 


% ---------------------------------------------------------
% Text
% ---------------------------------------------------------

\begin{document}
% \raggedbottom

% -----------------------------------------------------------------------------
% Review
% -----------------------------------------------------------------------------

\singlespacing
\begin{center}

{\Large \textbf{REVIEWS FOR:}}

\vspace{1cm}

{\Large \textbf{The visual environment and attention in decision making}}

\vspace{5mm}

% \textbf{authors...}

\vspace{1cm}
\end{center}


% -----
% Editor
% -----

\section{Editor}
\label{rev:editor}

\subsection{Overall evaluation}

% by using square brackets we can optionally create comment label that we can use later on
\com[com-editor]{Thank you for submitting "The visual environment, attention and decision making" for review and consideration for publication in Psychological Bulletin. The editorial board has completed its review. In addition to reading the manuscript myself, I was fortunate to have received reviews from three outstanding experts in the field. Their feedback is detailed and constructive, and I am grateful to them for their time and thoughtfulness.\\
\\
As you will see from the reviews, most of us noted that this work was conducted in a generally thorough manner and we all generally got a sense that it has some potential as a contribution to the literature. However, there are also several issues that require consideration. I will now summarize points that I viewed as central that were raised by the reviewers, provide some feedback of my own along the way, and conclude with my recommendation.
%
\begin{itemize}
    \item My impression was that Reviewer 1 was concerned that the ultimate impact of this work might be unclear because, in its current form, it would only capture the interest of researchers who are already aware of the issues it raises. As such, it remains possible that progress might occur without requiring the publication of this meta-analysis. In my view, this mitigates the potential contribution of this work and would require some efforts on your part to appeal to a broader audience.
    \item Reviewer 2 noted that there were issues with the structure of the paper that affected its readability. I agree that the method should come immediately after the introduction, as is standard for APA journals.
    \item Reviewer 2 also raised the need to discuss more deeply the role of visual factors to clarify this paper’s potential impact.
    \item Finally, Reviewer 3 raised a number of critical issues that would clarify the scope and theoretical contribution of this work.
\end{itemize}
%
Based on these brief highlights, I would expect that the reviewers’ comments can be addressed but some of them will require tight arguments, especially when addressing concerns relevant to the impact of this paper. Therefore, I invite you to submit a revision that addresses all the the points raised by the reviewers. Please let me know within a week if you will be sending us a revision by replying to this message. If you chose to resubmit, please do so within 3 months of receipt of this letter.}

We thank you and the reviewers for the comprehensive evaluation of our manuscript and the encouraging comments. We are grateful to you for pointing out the most important comments from the reviewers. In the revised manuscript we have now introduced line numbers and for each comment we refer to a line in the document where we have introduced a major change. To facilitate the revision we have also marked major changes in the main text with red color. 

We have provided thorough responses to each comment under the section for respective reviewer, but will briefly summarize the main points here. To the first point from Reviewer 1 concerning the impact we have revised the introduction so that it now addresses a broader audience from not only cognitive science, but also the disciplines economics, marketing, management, hospitality and transportation research. We have also written a longer section on the structure of the visual environment which should clarify the relevance of visual factors in representative design across all these disciplines. Regarding Reviewer 2's points, we have aligned the structure of the manuscript sections to APA standards and expanded the section on visual factors to better clarify the contribution of the manuscript. Regarding Reviewer 3, we have addressed all points including the request for access to less aggregate (raw) data. Providing this required an extensive recoding of all included studies and in view of the effort we decided to use the new data to strengthen the manuscript further. We have therefore included a new section on descriptive eye movements that provides basic information about average fixation likelihood and counts across all studies as well as intuitive results expressing effect sizes in these metrics instead of correlation coefficients.    

\com[com-editor-litsearch]{If you decide to prepare a revision, please also address the following technical points: The literature search was conducted over 2 years ago. You should consider updating the sample.}
    
We have now conducted an updated literature search, covering the period from 2018 to October 2020, which resulted in 11 new articles, bringing a total to 69. Corresponding section in Methods has been updated (see line~\lineref{litsearch} and line~\lineref{inclusion}) as well as the PRISMA flow figure (Figure~\ref{fig:flow_diagram}).
    
    
\com[com-editor-evaluation]{You state on page 9 that “some experiments operationalized more than one factor”, suggesting that not all your effect sizes were independent. If that is the case, multilevel analysis with robust variance estimates should be considered to handle the dependence in effect sizes and sampling variances. It is likely that the Hunter \& Schmidt correction could still be applied in this context (see Johnson et al, 1995: \url{https://doi.org/10.1037/0021-9010.80.1.94}). In any case, this point requires clarification and possible changes to your analysis.}

Thank you for this suggestion. We have implemented it in our analysis and now a ``sandwich'' estimator is used with a small-sample adjustment \citep{hedges2010} (see line~\lineref{sandwich-methods}). The results did not change substantially based on this change.


\com[com-editor-evaluation]{Like reviewer 2, I would like to see more clarity about the publication bias analysis. In addition, trim and fill is not a means to investigate the presence of a publication bias but more a way to correct for it when it is identified. “Eyeballing” the funnel plots as you did is not valid either. A more thorough approach should be considered. For a summary of available approaches, see Table 2 in van Aert et al. (2019; \url{https://journals.plos.org/plosone/article?id=10.1371/journal.pone.0215052}).}

Indeed, trim-and-fill only attempts to correct the effect sizes, not to identify whether publication bias exists. We also agree that we should be able to do better than just eye balling the funnel plots. As we explain in a response to reviewer 2 (see \ref{com-r2-publication-bias}), to identify the existence of a publication bias we have now included a PET-PEESE analysis and a new type of analysis based on the coding of studies citing public grants (see line~\lineref{PB-PET}). We explain these new analyses in details in the Methods section (see line~\lineref{pubbias-intro}). 


\com[com-editor-evaluation]{You should avoid presenting the same material both in tables and in the text to streamline the paper.}

We did our best to eliminate such redundancies throughout the Results section.


\com[com-editor-evaluation]{For the sake of reproducibility, please state in the paper which R package(s) you used in your data analyses.}

In Methods we already state which R packages we have used in our analyses, specifying the packages in a list of references:

\quotetext{Analyses were performed in R programming language with the help of several additional libraries \citep{R2020,datatable,tidyverse,metafor,irr,lme4,lmerTest,xtable,extrafont}.}

However, in our experience this is far from enough for reproducibility. This is why we are providing the data and all the code in an Open Science Repository (\url{https://osf.io/buk7p/}), readers with a bit of programming skills should have easy time reproducing all the analyses. We have added a reminder about this resource immediately after specifying libraries (see line~\lineref{OSF-pointer}). 


\com[com-editor-evaluation]{Also, please add a Public Significance Statement below the abstract in the masked manuscript file. \url{http://www.apa.org/pubs/authors/guidance.aspx}}

We have added a Public Significance Statement below the abstract. 


% -----
% Reviewer 1
% -----

\section{Reviewer 1}
\label{rev:r1}

\subsection{Overall evaluation}
\label{rev:r1sum}

\com[com-r1-evaluation]{The paper is concerned with the role that aspects of the visual environment (e.g., salience, position, surface size), as opposed to internal, cognitive factors (e.g., heuristics and biases), play in decision making. Past decision-making research that employed visual displays focused on cognitive processes that followed encoding of the experimental stimuli, and ignored possible effects driven by features of the visual environment. Though this strategy no doubt simplified the problem space for these researchers by, in effect, controlling the broader context of stimulus input, the authors claim that ignoring visual features of the environment that are inevitably present in real-work circumstances has compromised the external validity of such work. The authors present a meta-analysis to support their claim that complete models of attention and decision making must include some accounting of the visual environment, and the work of the eyes.\\
\\
There's much to like in the approach the authors are taking. In effect, they are calling out the decision-making field for ignoring the world of bottom-up visual processes in favor of a focus on the (perhaps) more interesting top-down cognitive processes. That is, decision making theorists have taken an "all-things-being-equal" attitude toward the visual environment from which decision-related stimuli are sampled, but in the real world all visual scenes are not equal, so decision theorists need to model eye movements, stimulus-driven capture of attention, and other vision-based variables if they want complete models. The overall contribution of the paper is as a call for the field to integrate cognitive and visual processes in models of attention and decision making.} 

We thank you for a concise summary and we are very happy that you liked our approach.  We are grateful for suggestions for improving the manuscript, we address each of them below. In the revised manuscript we have now introduced line numbers and for each comment we refer to a line in the document where we have introduced a major change. To facilitate the revision we have also marked major changes in the main text with red color.


\subsection{Comments}

\com[com-r1-praise]{The structural aspects of the meta-analysis are sound. The effects are clear, if not impressive. The paper is well-written and should be accessible to a non-specialist audience.}

Thank you for the feedback, in particular it is good to know that our writing strikes a right balance. Regarding effect sizes please note that they have changed in the revision. We updated the literature search and included 11 new articles. Some effect sizes did became larger (e.g. salience), but the most important point is that when taking into account publication bias we now see that visual factors play a larger role than cognitive factors. 


\com[com-r1-field-divisions]{Overall, however, I'm somewhat mixed about this paper. On the plus side, the authors have revealed that an important branch of cognitive science has overlooked systematic sources of variance that could be - and should be - modeled. Moreover, their analysis yields some fairly specific directions for future research. On the minus side, it's not clear what the ultimate impact of this paper will be. In many cognitive departments vision science anchors one end of the processing spectrum while higher-level areas such as decision-making anchor a distant, opposite end. Those who are most likely to respond to this call are those already aware of the situation. Which leads to a related concern: Is this call, in fact, needed for progress to be made? While reading this ms I found myself occasionally thinking "Right - well then why don't you go ahead and just do the studies that you claim should be done?"}

We are very pleased that you share our view of the need to integrate visual factors in decision research, but we remain sceptical that progress in this direction is likely to happen by itself. You ask why we do not just go ahead and do the required work. We are indeed preparing work with the aim of integrating cognitive and visual processes in models of attention and decision making. Even with a generous assumption that other teams working at this intersection are working or planning to work on similar ideas, we believe this will not be sufficient to achieve such integration. Theoretical challenges and model development look daunting at the moment, and it is highly unclear whether a handful of teams can make a serious dent in a reasonable time frame. Hence, our call to action needs to reach broader audience of vision scientists and decision making scholars. The reach of the issues we raise goes beyond vision science and decision making, however -- attention plays an important role in theories outside of these two domains. We acknowledge that this was not clear in the initial manuscript and have added a paragraph reviewing theories on attention and decision making from other disciplines (see line~\lineref{r222}). We show that disciplines such as economics, marketing, management and organizational research, resource economics, hospitality and transportation research all have well developed concepts of attention and decision making, and that in all cases attention is strictly seen as a top-down cognitive process. We hope that our call to action will result in diverse set of contributions from all of the disciplines.\\

Admittedly, many decision making scholars tend to view visual factors as nuisance variables that rational decision makers (should) ignore. This can also be seen in a comment made by reviewer 3 (see comment \ref{com-r3-salience}): 

\quotetext{One problems with salience in the context of choice is that people deliberately try to counteract it since they might become aware of the manipulation intention, or they want to be rational and look at all information (at least once)}. 

%In our opinion, the comment does not hold for all studies, mainly for simple standard paradigms with few options (risky choice, strategic choice, intertemporal choice etc.), but still, these paradigms account for a large portion of decision making research.
To reduce the chances that decision making researchers ignore our results and call to action, we opted for providing additional arguments why visual factors should not be viewed as nuisance variables. In a new section on the visual environment we have explored the relation between the visual and cognitive factors (starting at line~\lineref{vis-env}). We show that visual factors in natural environments are highly correlated with decision relevant cognitive factors, e.g. salience, size, and position are ecologically valid predictors of product attributes, product popularity and even of product price. When experimentally eliminating or randomizing away visual factors, as is usual in simple standard paradigms in decision making, the experimenter removes any ecological validity of visual factors, and therefore, it is not surprising that participants in these experiments would ignore salience (or any other visual factor). With these arguments and with the effect sizes of visual factors, we hope to, at the very least, make decision making researchers more cautious when generalizing results of their models to real world (where environments are likely to abound with visual factors and substantially differ from a standard lab setting), if not contribute directly to the integration efforts (see  line~\lineref{call-to-action}).


\com[com-r1-acceptance]{There are reasonable responses to these questions, of course, and overall, the strengths of this ms outweigh the weaknesses. For those reasons I'm favorably inclined toward publication.}

Thank you for your positive evaluation, it is useful to know that we are on the right track.



% -----
% Reviewer 2
% -----

\section{Reviewer 2}
\label{rev:r2}

\subsection{Overall evaluation}
\label{rev:r2sum}

\com[com-r2-evaluation]{Thank you for submitting your paper on the role of visual attention in decision-making processes. I read the paper with a great interest. I agree with the authors that it is important to gather more insights into the role of visual attention in decision making and better understand the different factors that drive attention, particularly across disciplines. There are however some elements of the paper that could be further improved and some points that require clarification. Below I discuss major problems and auxiliary issues.} 

Thank you for your interest as well as for comments for improving the manuscript. We address each comment below. Please note that in the revised manuscript we introduced line numbers and for each comment we refer to a line in the document where we introduced a major change. To facilitate the revision we also marked major changes in the main text with red color.


\subsection{Comments}

\com[com-r2-methods-new-location]{First, the structure of the paper is quite surprising and makes following the authors' reasoning difficult. Including the method after the results and discussion forces the reader to search for information needed to understand what the authors did. Please change that.}

We have made the requested change, the method section now appears after the introduction, starting at line~\lineref{methods-new-location}.


\com[com-r2-discuss-visual-factors]{I also think that the authors should elaborate on the theoretical framework behind their study. While the authors provided a rather thorough literature review in the introduction, I missed a deeper discussion on the effects of visual factors, which would substantiate the relevance of the study.}

There are two paragraphs in the introduction that elaborate on the visual factors (starting at line~\lineref{vis-env}). Here we correct some of the issues you raise below and now we have strengthened the theoretical framework with a new third paragraph about the relation between visual and cognitive factors and how we expect that visual factors come to play a role in attention (starting at line~\lineref{natural-scene}). A short summary: We review literature from natural scene statistics to explain how visual factors come to play a role in attention from an evolutionary perspective, we also review studies on natural scene statistics in decision environments and have found very high correlations between visual factors like salience, surface size, and position, and decision relevant factors such as price, product popularity or product attributes. These ecological correlations probably makes visual factors efficient for guiding attention to relevant information and could explain why decision makers readily use visual factors.   


\com[com-r2-visual-elaborate]{I believe that the authors should elaborate on the visual factors more and earlier in the manuscript. I think 'silence' should be better defined, especially since the authors say in the discussion that "there are potentially other less researched visual factors not covered in our study that influence eye movements, e.g. motion or sudden onsets are known to capture eye movements involuntarily," but at the same time they claim that motion was coded as an element of silence: "We coded studies as salience if they operationalized one or more of the known dimensions of salience such as color, edge density, contrast, or motion."}

Thank you for raising these points. We have provided an operational definition of visual factors at the very beginning of the manuscript (see line~\lineref{visfac-def}). We have then added a more clear-cut definition of salience pointing out that it combines several low-level visual factors such as color, contrast, edge density, and motion (see line~\lineref{salience-def}). Finally, we have revised the sentence in the discussion which creates confusion about the operational definition of salience and the included factors (see line~\lineref{other-factors}). We have now pointed to possible visual factors that are not included in the definition of salience such as factors studied in natural scene statistics, light and shade, texture, or occlusion by objects, gestalt principles such as the laws of proximity, similarity, closure etc., or overall image properties such as feature or design complexity or visual clutter. 


\com[com-r2-visual-complexity]{Relatedly, it is not clear what the authors mean by 'visual complexity,' which seems to be more than the number of alternatives (see e.g. Pieters, Wedel, \& Batra, 2010).}

We have removed the sentence about visual complexity from the discussion of set size. As you point out, the explanation of these concepts was not clear. After considering this further, we believe that it does not follow that increasing the set size must increase visual clutter. If we are increasing the set size with objects that are similar in visual features then visual clutter should not change e.g., a notion similar to the concept "heterogeneity of brand background" in Pieters, Wedel, \& Batra, 2010. We therefore discuss visual clutter, feature complexity and design complexity as unexplored visual factors in the context of decision making (see line~\lineref{other-factors}).  


\com[com-r2-relation-salience-other-factors]{I think that when defying the studied factors, the authors should also explain what the relationship between silence and the other factors, such as surface size, is.}

We believe your question deserves a two-fold answer. On one hand, it is possible to operationalize visual factors independently of each other i.e., one can manipulate salience without influencing surface size, set size, or position. This means the visual factors can be made independent of each other in experimental designs. We have added this explanation to line~\lineref{independence}. On the other hand, in natural environments the visual factors are likely to be correlated with each other since we know that they correlate with common cognitive factors. For instance, popular products are more likely to have larger surface areas (number of facings) and also more likely to be positioned on top shelves in the supermarket. We have added this explanation in the new section describing the natural visual environment (see line~\lineref{dependence}). 


\com[com-r2-why-these-factors]{Finally, it is not entirely clear why the authors decided to study the chosen factors - that should be justified, especially since the authors suggest there are more relevant factors that can attract attention and affect decisions.}

Thank you for pointing this out. The reason for including the four visual factors is actually straightforward: as far as we know, these are the only visual factors studied in the context of decision making. We have stressed this at line ~\lineref{studiedfactors}. The cognitive factors are fairly broad in their definition and therefore capture most of the literature. We have explained this in the section dedicated to study approach at line~\lineref{factorinclusion}.


\com[com-r2-operationalize-decision-making]{In a similar vein, the authors should operationalize decision making at an earlier stage to help the reader understand the research inclusion criteria.}

We have moved the method section up so it appears before the results section (see line~\lineref{methods-new-location}). We hope this will help the reader to understand how we operationalize decision making and the inclusion criteria.  


\com[com-r2-publication-bias]{Could you also discuss the publication bias a) before presenting the results and b) more in depth? Why it occurs in this type of studies and what the plausible consequences are, as well as how the adjustment improved the analysis?}

We have now discussed the publication bias analysis in the method section which appears before the result section (see line~\lineref{pubbias-intro}). We hope this provides the necessary background for appreciating the analysis results. We have also extended both the analyses and the discussion of them as you propose. We have for instance, included a PET-PEESE analysis and new type of analysis based on the coding of studies citing public grants. We have extended the presentation of the publication bias results with these new analyses, using them to first verify that publication bias exists, before moving forward to the trim-and-fill correction procedure (see line~\lineref{PB-PET}). In the results section, the publication bias analyses appear after the main and moderator analyses since this is the most canonical order of presentation.


\com[com-r2-metric-correction]{The authors noted that there are many discrepancies in the literature when it comes to analyzing and reporting eye-tracking data. I agree. Researchers focus on different metrics. Also, since eye-tracking data are often skewed and zero inflated, different authors introduce different methods of analysis but also various transformations (such as logs, etc.), hence, the reported results are difficult to compare. How did the authors account for that? Was the introduced correction enough?}

As you point out, there is a wealth of different metrics and ways of handling eye tracking data. Fortunately, the vast majority of studies we identified reported results in either fixation likelihood, fixation count, total dwell time, or dwell count. Our section on multiple metrics in Methods section shows that effect sizes computed based on these four metrics are comparable with some adjustments. This is probably not true for other metrics e.g., we do not expect that time to first fixation, first dwell duration, dwell time proportion etc etc provide similar results. A few studies reported results in metrics other than the four included ones and consequently we had to exclude these. The answer to your question, therefore, is that not all metrics are comparable, but some metrics are after certain adjustments, and we focus only on the comparable metrics. To make the relation between effect sizes and underlying eye tracking metrics more transparent we have coded descriptive eye tracking metrics for all included studies. We have then written a new section (starting at line~\lineref{descriptiveEM-overview}) which shows how effect sizes are extracted from descriptive eye movement metrics and can be transformed back into absolute changes in fixation likelihood, fixation count, and total dwell time. This section also provides a more intuitive interpretation of the effect sizes. 


\com[com-r2-search-language]{Can you really claim that no restrictions on language were imposed, since the search terms were in English?}

You are right, we cannot make this claim and we have therefore deleted it from the section on literature search (see line~\lineref{litsearch}). 


\com[com-r2-socio-demographic]{On p 24. The authors state that they "excluded studies where participants were selected based on clinical diagnosis or specific socio-demographic traits e.g., visual disorders, age-related visual diseases, age restrictions such as adolescents or infants." What would be the reasons for removing certain age groups and what are the possible consequences? Were the samples that were not removed representative or were these mostly student samples? Could you share more information about the studied populations?}

The reason for these exclusion criteria was decided a priori since, for instance, participants with clinical diagnoses usually have very different eye movement patterns from non-clinical participants. We actually only excluded 3 studies on this account -- these were studies involving children.


\com[com-r2-focus-on-count]{I am also curious why the authors decided to focus on count measures rather than time, such as dwell time? The authors explain why they did not want fixation duration. Is it due to the ease of interpretation of count variables?}

We have expanded the section on multiple metrics to explain our reasons more clearly (see line~\lineref{metrics}). A brief summary of the revised paragraph is that both total dwell time and dwell count are composite metrics of fixation count with either fixation duration or inter-AOI transition count. This composite nature make them less construct valid in terms of the included visual and cognitive factors since fixation duration and inter-AOI transition count are diffusely related to the included visual and cognitive factors. Fixation likelihood is a discretization of fixation count, which leads to loss of information and underestimation when fixation counts are above 1 for all AOI's. In the end, fixation count is the only metric left that is easy to interpret, construct valid and will not underestimating effects due to discretization. 


\com[com-r2-low-icc]{Could you explain where such low ICC for the effect size comes from?}

We believe the reason for the low ICC for effect sizes is a combination of two factors: (1) the initial round of effect size coding was performed by a junior researcher, and (2) as you note above, eye tracking studies are very complex, often relying on multiple metrics, transformations etc. In the process of coding the descriptive eye tracking data the senior scientist responsible for this task also recoded effect sizes (blinded to the initial data). Testing the agreement between the original data and the recoded effect sizes shows that ICC is very high, ICC = .923, indicating a reliable data set. In revising the ICC we also realized that by mistake several variables were represented as ICC's even though they were kappas (see line~\lineref{reliability}).


\com[com-r2-final]{I hope my comments will help the authors move their work forward.}

We are grateful for your suggestions, they have definitely helped us to improve the manuscript.



% -----
% Reviewer 3
% -----

\section{Reviewer 3}
\label{rev:r3}

\subsection{Overall evaluation}
\label{rev:r3sum}

\com[com-r3-evaluation]{In the paper, the authors present a meta-analysis on factors influencing attention in decision making. The paper is overall very well done, makes several important methodological contributions and is very well written. Technically, this paper is also extremely sophisticated and I did not spot any substantial weakness concerning methodology. Overall, the paper has the potential to make a substantial contribution to the field and to be published in Psychological Bulletin after same revisions. In the revision, several issues should be addressed that mainly concern issues of clarification, presentation and discussion.} 

We thank you for your positive evaluation and suggestions for improving the manuscript. We address each comment below. Please note that in the revised manuscript we introduced line numbers and for each comment we refer to a line in the document where we introduced a major change. To facilitate the revision we also marked major changes in the main text with red color.


\subsection{Comments}

\com[com-r3-JDM-doesnt-care-about-attention]{The title and some other parts of the text are slightly misleading in that the induce higher expectations to which the paper cannot fully live up. It should be more clearly stated that the investigation is (only) concerned with effects on attention in choices and not with the - from a decision theoretic perspective - perhaps more important link between attention and choice. This also should be acknowledged in the discussion and other parts of the paper might be toned down a bit (e.g. policy significance statement). Most of the mentioned models (and other most prominent models in choice) do not really care overly much about attention and their predictions for it. They want to predict choices, sometimes also choice processes - attention is sometimes used to test process assumptions. This should be clarified. Also it should be acknowledged that from all we know so far the causal effects of attention on choice are tiny to not existing - at least if the studies use proper controls (e.g., Ghaffari \& Fiedler, 2018). It is still important to understand attention in choice processes but it also has to be stated that we can predict choice very well and close to choice reliability by our standard models without taking any attention effects into account (e.g., Glöckner \& Pachur, 2012; Glöckner, Hilbig \& Jekel, 2014). In the eye-tracking study contained in Glöckner et al. (2014), for example, there was a huge effect of environment on choice but no effect on attention. Hence, attention seems to often not even mediate effects of context factors on choice. This should be acknowledged as well. Still, it is certainly important to understand effects on attention in choice processes but these effects have to be discussed more in relation to the core aims of the models (predicting choice).}

Thank you for these comments. To keep a better overview, we answer them one at a time in the list below. 

1. We have changed the title to ``The visual environment and attention in decision making'', this makes it more precise that we are concerned with visual environment and attention in the context of decision making. We have also toned down our abstract and significance statement.

2. We now acknowledge at the very beginning of the manuscript that most decision research is concerned with predicting choices (see line~\lineref{r31}) and sometimes also choice processes, and that attention is then used to test process assumptions (see line~\lineref{r31b}).

3. Rather than open a discussion about the causal effect of attention on choice, which is outside the scope of the manuscript, we have decided to remove the claim about causal effects in the introduction (see line~\lineref{r32}).

4. We agree that attention is not necessarily useful for predicting choices. We have added a paragraph where we point this out and try to provide a few explanations why this could be the case (see line~\lineref{r33}). We would like to make two points, however. First, ability to predict choices really depends on a task -- accuracy can become fairly poor already in fairly standard multi-attribute choices. If we go slightly outside of classical lottery type of tasks and consider reinforcement learning tasks (which we, admittedly, do not consider here), taking into account attention can matter a lot. For example, \cite{stojic2020uncertainty} shows that with attention accuracy of predicting choices in multi-armed bandit tasks almost doubles. Second, your point serves very well to illustrate the point about the role of the visual environment. Our interpretation is that in standard JDM tasks in the lab the visual environment has no ecological validity in terms of relevant decision criteria (validity etc), in fact, visual factors are often explicitly controlled for. This could explain why decision makers might ignore visual factors in such tasks. \\


\com[com-r3-salience]{One problems with salience in the context of choice is that people deliberately try to counteract it since they might become aware of the manipulation intention, or they want to be rational and look at all information (at least once). At least in the simple standard paradigms, people mainly look up all information and then show double checking, which not necessarily have to do with the core of the decision process. It should be acknowledged that (a) it is therefore not surprising that effects of salience on attention are smaller in this context than in purely visual paradigms; (b) that people mainly look up all pieces of information at least once (the authors could provide a measure for this: the average dichotomous fixation rates (yes / no)). When just looking at the total number of fixations as it is done in the current analysis, (c) it is also not entirely clear what the attention tells us about the process since it often involves double checking and qualitatively different kinds of processes - sometimes involving long fixations indicating deliberate processes and sometimes shorter ones indicating scanning and double-checking even under typical deliberation instructions \citep[e.g.][]{horstmann2009}. Of course, the authors show a high correlation between various measures but it should be acknowledged that important process details are most likely missed.}

We have addressed your points in the following ways:\\
a) We have now discussed the possible effects of salience and visual factors in simple standard JDM paradigms and why salience may not have an effect here (see line~\lineref{r33}).\\ 
b) We have also discussed that decision makers are likely to attend to all information in these simple paradigms (also at line~\lineref{r33}). Your point about the dichotomous fixation rate took a good deal more space to clarify. First, we had to go back to the data and code descriptive statistics for all included studies. Based on this we wrote a new section on the descriptive eye movement statistics. The answer to your point is actually very surprising. It turns out that across all included studies providing descriptive statistics, the average dichotomous fixation rate (we refer to fixation likelihood) is only 61\% (see line~\lineref{descriptiveEM-overview}). So even though it may be true that fixation rates are nearly 100\% in studies with simple paradigms such as risky gambles, this is certainly not true in general of decision tasks.\\ 
c) We agree that important details may have been missed due to the selection of the dependent variables. However, many of these important metrics such as fixation duration and scanpath metrics such as the SI are reported in very few papers and do not lend themselves to a meta-analysis. We have added this paragraph to the discussion (see line~\lineref{r422c}). Regarding re-fixations, whether they should be classified as double checking, depends on precise setup and theoretical framework -- for example, modelling the decision process with sequential sampling model re-fixations are in fact part of the evidence accumulation, and hence speak directly about the process \citep[e.g.][]{krajbich2010a}. 


\com[com-r3-left-right]{The left vs. right distinction is a bit oversimplified since these effects mainly concern temporal dynamics and not an overall effect: people typically start reading left and then move to right and later to the emerging favored option (e.g., Fiedler \& Glöckner, 2011). Hence, a null effect is not overly surprising. Arguably, sometimes primacy or coherence effects influence choice processes and therefore the gaze-cascade effect might lead to a left looking bias overall - as found here. It is ok to report the effect as it is now but it should be acknowledged that this is a simplification and that there are reasonable process assumptions that might explain these effects relatively easily also within established models.}

We agree with your thoughts on the left vs right factor and have mentioned in the results that despite our null effect there might be temporal dynamics effect which are not covered in this meta-analysis (see line~\lineref{r423}). 


\com[com-r3-set-size]{I did not fully understand what the set size manipulation effect size means. This should be better explained. Describing an example study would be helpful - also for the manipulation of central position. Is it that fixations to EACH option reduces if more options are shown simultaneously on the screen? Since set size is numerical this could also be analyzed using absolute numbers of options (was this done here or just compared small vs. large?) -it would be great to give a bit more details; also to understand why for example Orquin found no effects at all whereas others found huge effects. I expect this not to be random.}

We have added more details to the method description of each independent variable (starting at line~\lineref{r424a}) to explain what the effect direction indicates. For set size we have also provided an example (see line~\lineref{r424d}). Basically, your interpretation is correct, a positive effect size indicates that fixations to each option are reduced with larger set sizes by comparing small vs large sets. We do not compare sets based on absolute numbers since some studies operationalize set size in number of attributes and some in number of alternatives. When split by this alternative vs attribute moderator, there are not enough studies to meaningfully perform a meta-regression, but we do compare the effect sizes when split by this factor. Future studies could examine this question in more depth using our descriptive data for the studies reporting an overall fixation likelihood. However, this would require coding the set size - a task which may turn out to be infeasible since not all authors report sufficient details about properties of the stimuli, definition of AOI's etc. 


\com[com-r3-descriptive-data]{I highly appreciate that the data is shared, but the provided data is on a very high level aggregated and descriptions for the variables are missing (code book). In the current form, the data cannot really be used for further questions that researchers (like I) might have to understand what is going on. This should be improved. If available less aggregated or if possible even raw data should be provided.}

Following your suggestion we have coded raw data where reported. More specifically, we have extracted raw eye movement measures that underlie the effect sizes -- fixation likelihood, count and dwell time per condition per study (see a new section in Methods, beginning at line~\lineref{descriptiveEM-coding}). Hopefully this extended data set will allow other researchers to reuse our data more easily.\\

We also realised that we could use the new data to improve our understanding of the results. Details on this analysis can be found in a new section in Methods, beginning at line~\lineref{descriptiveEM-analysis}, and we have reported these results in a new section in Results, beginning at line~\lineref{descriptiveEM-overview}.
First, we have provided simple distribution plots and averages which allow the reader to get a sense of how the raw data is associated with the effect sizes (Figure~\ref{fig:em_figure}, panels A-C). For example, one can see that there is a lot of variance in fixation likelihood across studies -- in some studies participants fixate nearly all AOI's, but there is also a large number of studies in which participants fixate half or less of the AOI's. Second, we transform the synthesized effect sizes for each independent variable into its corresponding effect on fixation likelihood, fixation count and total dwell time. We achieve this by regressing descriptive measures (appropriately transformed) on effect size correlations, using a linear mixed model with a random intercept, grouped by article to account for correlated errors. Overall, all three measures strongly correlate with the (transformed) effect sizes, giving us confidence for converting effect sizes into original measures using the fitted models. Table~\ref{tab:em_results} reports results which the reader can use to more intuitively compare the effect sizes for each independent variable in terms of the equivalent effect on fixation likelihood, fixation count, and total dwell time.\\

Finally, we have provided a code book in the Open source framework repository, alongside the data file -- thank you for spotting this omission.


\com[com-r3-choice-bias]{I found it unfortunate, that the gaze cascade effect is renamed in choice bias. This implies - due to conventional use - that there is a bias in choice, but this is not what is meant by it in the current context. The authors might consider using a different term or at least to add in the main tables and figures are reference to the classic term. Recently, a process model with an even more detailed prediction concerning information search, the attraction search effect, has been proposed and empirically validated (Jekel et al., 2018). This should be briefly discussed as well in the GD as current model development.}

We agree that the name ``choice bias'' is potentially misleading and now refer to this factor as ``choice-gaze effect''. We have integrated a paragraph of the PCS model and the interesting attraction search effect in the discussion. We also suggest that the model and the implication for attraction search may potentially provide the necessary background for explaining the effect of preferential viewing  as well (see line~\lineref{r426}).  

\com[com-r3-interactive-activation]{I was missing in the GD a brief discussion concerning the link between models for perception (e.g. McClelland / Biederman / Kintsch etc.) and choice - mentioning classic interactive activation models. Some models try to bridge the gap by using essentially the same processes in both contexts, while other assume very different processes in higher cognition (i.e. decision making). A brief discussion would be warranted given the broad readership of this journal.}

We agree that this is a interesting and promising avenue and have added a paragraph to the discussion about the relevance of bridging the perceptual and choice domain (see line~\lineref{r427}). 

\clearpage


% -------------------------------------------------------
% Introduction
% -------------------------------------------------------

% part necessary to accommodate differences between review section and main text
\doublespacing
\setcounter{page}{1}
\setcounter{secnumdepth}{0}

% original part
\linenumbers
\maketitle

\section{\normalfont\normalsize Public significance statement}
\noindent Visual attention in decision making is influenced by the relevance of information and by how information is presented i.e., visual factors like the position, salience, and size of information and the number of competing information elements. We show that \textit{how} information is presented is more important to attention than the relevance of the information. Policy makers and companies can leverage visual factors to mislead or guide our attention to information that enhances welfare supporting behaviors.\par
\newpage

% -----------------------------------------------------------
% Introduction
% -----------------------------------------------------------
\section{Introduction}

% \section{Motivation and Problem}

Many of our decisions are made in environments where the relevant information must be acquired visually. In such visual environments choice alternatives can differ in their position, surface size, salience and many other visual properties. Consider, for instance, encountering a product with a surprising color on a supermarket shelf, or a restaurant menu where certain items take a prominent position and perhaps have an accompanying picture. Such visual properties have all been shown to influence our attention \citep{corbetta2002a,borji2012a,dehaene2003a,clarke2014a, rosenholtz2007a} and governments and companies are becoming more aware of how to use these visual properties to communicate effectively with citizens and consumers \citep{orquinwedel2020}. There is growing evidence showing that attention plays an important role in decision making \citep{gidloef2017a,krajbich2010a, stojic2020uncertainty, callaway2019a, gluth2018, gluth2020}, and can even causally affect choices \citep{ghaffari2018a, paernamets2015a, shimojo2003a}. However, the role of visual environment factors is almost completely absent from prominent decision theories. In most theories, cognitive factors such as goals in the decision task determine the relevance of objects and, either explicitly or implicitly, whether and when we look at them. Here, we ask whether decision research is building on correct assumptions about visual attention and the role of the visual environment, and provide an empirical assay of the relative importance of various visual and cognitive factors to guide theory development and real-world applications of visual factors.\\

% \section{Decision research (mostly) ignores bottom-up factors}
Most decision research considers attention to be determined by the decision process, that it is driven by the goal relevance of objects rather than their visual properties. In many prominent decision making models this assumption is implicit. Consider, for example, the prospect theory model of how probabilities and values of choice alternatives are integrated to arrive at a preferential choice \citep{tversky1979}. Alternatives are treated equally according to this model, and nothing in the model indicates that one piece of information should attract more attention than other. Prospect theory and related variants of expected utility theory focus on capturing the final choice, not the process of how people arrive at the choice. However, popular process-oriented decision models commit to similar assumptions about attention. Consider, for example, satisficing, elimination-by-aspect, or the lexicographic heuristics \citep{payne1988, simon1956a}. While these models all specify different information search processes, they make similar implicit assumptions about the nature of visual search and hence attention in decision making. The  models assume that information search is determined by a search rule inherent to the decision process, e.g. attend to alternatives one at a time until a satisfactory alternative is found \citep{stuttgen2012}, or attend to information cues in order of their predefined validity until a cue is found that identifies the best alternative \citep{krefeld-schwalb2019a}.\\ 

In recent sequential sampling models of decision making attention has had a more explicit role. Sequential sampling models assume that stochastic evidence for an alternative is accumulated over time and when the integrated evidence reaches a threshold a choice is made. This is a process-oriented model that aims to capture how people balance the value of accumulating more information with the cost of taking more time to reach a decision \citep{forstmann2016}. In two influential variants of these models attention plays an important role, by determining how evidence is sampled in favor of choice alternatives \citep{busemeyer1992} or by determining the weight assigned to the evidence \citep{krajbich2010a, thomas2019}. In these models, attention fluctuates randomly between choice alternatives or choice attributes until a choice is made. The implicit assumption being, that in the long run attention is uniformly distributed over alternatives and attributes. This is a stochastic equivalent to a maximizing decision rule such as the weighted additive which assumes that a decision maker attends equally to all information \cite{gloeckner2011a, payne1988}. In other words, even though attention exerts an influence on choice, this influence is random and neither controlled by goals nor the visual environment. Recently, sequential sampling models have been proposed in which attention is guided by the value of choice alternatives \citep{callaway2019a, gluth2018, gluth2020}. This assumption is supported by empirical findings demonstrating value based attentional capture, i.e. the effect that objects associated with rewards capture attention \citep{lepelley2015}. The models are reminiscent of an earlier idea by \cite{shimojo2003a} who proposed that decision makers attend preferentially to high value alternatives, which increases their value further, thus creating a feedback loop and increasing likelihood of gazing at the ultimately chosen alternative.\\ 

A few studies have proposed decision models where attention is not driven only by the goal relevance of alternatives, but also by their visual properties, focusing on salience, i.e. the visual conspicuousness of a stimulus relative to its surroundings. For example, \cite{towal2013a} showed that salience continuously influences the decision process by making some choice alternatives more likely to attract fixations, but it does not influence the drift rate, i.e. the speed of accumulating evidence, towards salient choice alternatives directly. \cite{chen2013} provided evidence that salience can determine the onset of drift towards a choice alternative, but not the drift rate itself. Finally, \cite{navalpakkam2010} showed that decision makers in a reward harvesting task made choices by combining value and salience, consistent with an ideal Bayesian observer. This work suggests that salience can also influence the decision process directly and not merely by biasing attention.\\ 

% \section{Why do we need bottom up factors in decision making models?}

The common assumption about cognitive factors being the only or main factor driving attention in decision making is inconsistent with a number of findings. \cite{vanderlans2008}, for instance, find that 2/3 of variance in attention is due to factors in the visual environment, unrelated to the decision task, and \cite{towal2013a} find that 1/3 of variance is due to visual factors. There are also several model free studies showing comparative effects of cognitive and visual factors on attention in decision making \citep{gidloef2017a, orquin2015a, orquin2019a}. Moreover, there is evidence that the visual environment influences choices by biasing visual attention. For instance, decision irrelevant visual factors have been shown to influence choices by changing the amount of gaze \citep{peschel2019, chandon2009a} or the order of gaze \citep{reeck2017a}. Even studies examining purely cognitive models of decision making often implicitly acknowledge the influence of visual factors by taking great effort to eliminate them by controlling the size, position, and salience of information \citep{brandstatter2014, gloeckner2011a, perkovic2018}.\\

Further evidence for the role of visual factors comes from vision science. The few studies that modelled the influence of the visual environment on attention in decision making focused exclusively on salience \citep{chen2013,navalpakkam2010, towal2013a}. This focus seems justified - a great deal of research in vision science has concentrated on salience, for a review see \cite{borji2012a}. The term salience refers to stimuli that differ from their surroundings in terms of visual conspicuity and it has been shown that observers are more likely to gaze at stimuli that are high in salience \cite{itti2000}. However, there has been much debate about the role of salience in guiding attention some arguing that it plays no role in, for instance, real-world behavior \citep{tatler2011a}. Besides salience, there are at least three other visual factors that are likely to guide attention in decision making \citep{orquin2013a, wedel2008}.\\

One factor is the relative surface size of stimuli, which refers to the proportion of the visual environment occupied by the stimulus \citep[for a review see][]{peschel2013a}. Increasing the surface size of choice alternatives has been shown to increase gaze by up to 25 \% \citep{chandon2009a}. Increments to surface size exhibit a diminishing marginal effect on eye movements \citep{lohse1997a}. A second factor is the position of stimuli which has been shown to influence eye movements and is sometimes corrected for in vision research models when estimating the influence of other variables of interest \citep{clarke2014a}. In a decision context alternatives are normally placed in different spatial locations, which means that position effects like left-to-right (reading) direction and centrality are likely to influence eye movements and choices \citep{atalay2012a, meissner2016a}. A third factor is the set size which in a decision context normally is operationalized as the number of alternatives or attributes. Increasing the set size generally slows reaction times to identify search targets \citep{wolfe2010} and may also increase the visual complexity by the addition of more and different visual stimuli. Visual complexity has been shown to increase the difficulty and amount of visual search \citep{rosenholtz2007a}. An important point about these visual factors is that all four are likely to vary in natural environments and have been shown to affect attention simultaneously \citep{orquin2019a}. While decision research often sees the visual environment as a nuisance factor and try to eliminate its influence on decision making \citep{brandstatter2014, gloeckner2011a, perkovic2018}, companies and governments often use the same factors to compete for the attention of consumers and citizens \citep{pieters2017, orquinwedel2020}.\\   

Despite these findings on the presumed importance of visual factors in attention and decision making, they have had only a small impact on theory development. While attention and its cognitive antecedents recently started playing a prominent role in decision theories \citep{callaway2019a, gluth2018, gluth2020, krajbich2010a, noguchi2018, thomas2019, usher2019}, the role of visual factors has been largely ignored. There are only a few studies that have proposed and tested models that incorporate the influence of the visual environment on attention in decision making \citep{chen2013, navalpakkam2010, towal2013a}. Moreover, these studies have focused exclusively on salience, despite the other visual factors that are likely to be relevant as well and their joint contribution. A systematic review that provides evidence on how important visual factors are individually, as well as relative to cognitive factors, would give a new impetus to theory development and real-world applications incorporating the role of the visual environment; or justify the lack of it. The increasing availability of eye-tracking equipment has paved the way for such a review. Eye-tracking provides a way to unobtrusively measure the influence of both visual and cognitive factors on attention in decision tasks. In the last two decades numerous model free eye-tracking studies appeared, situated in a decision context. These studies span many disciplines, from behavioural economics and consumer psychology to cognitive psychology, computational neuroscience and vision science, which potentially explains why such a review has not been done before.\\

% \section{Study approach} 

Here, we assess the importance of the visual environment in decision making by empirically examining the magnitude of effects of various visual factors on attention in decision making and comparing them with cognitive factors. We focus on four visual factors -- salience, position, surface size and set size -- and three cognitive factors -- task instruction effects, preferential viewing and choice bias. We collect effect sizes from studies on eye movements in decision making and meta-analyze them to get reliable effect estimates. To do so, we develop new methods to address methodological challenges of meta-analysing eye movement data. Our findings show that among the visual factors positioning a stimulus in the centre of the field of view has the largest effect, while salience has the smallest effect on attention. Relative to cognitive factors, visual factors have somewhat smaller effects on eye movements. However, since all visual factors can influence attention simultaneously, in cases with multiple factors \citep{gidloef2017a, orquin2019a}, these could jointly have a larger influence than cognitive factors. Overall, these results show that characteristics of the visual environment have reliable effects on eye movements in decision making and that the effects are present across various decision contexts and tasks. This suggests that future theories and models of decision making should integrate visual factors directly rather than see them as nuisance factors. Governments and companies can effectively guide decision makers' attention to information by positioning it centrally, by making it larger, by reducing the set size (competing information), and perhaps to some extent by making it more salient.  
% -----------------------------------------------------------
% Method
% -----------------------------------------------------------

\section{Method}


% ---------------------------------------
\subsection{Literature search}
% ---------------------------------------

Literature was searched in two rounds: first round covered all the literature until March 2018 and second round covered literature published from 2018 to October 2020. Web of Science was searched using the following terms: eye track* OR eye move* OR eye fix* AND decision making OR choice. Gray literature, such as reports and unpublished work, was identified in the first 1,000 hits on Google Scholar (200 hits in the second round). No restrictions on publication date were imposed. Additional literature was identified by searching the reference lists of the identified papers and through contact with the authors. Calls for unpublished studies were distributed to the relevant research communities via the following email lists; European Association for Decision Making (EADM), Society for Judgment and Decision Making (SJDM), and European Group of Process Tracing Studies (EGPROC). The search resulted in 412 studies screened for eligibility.


% ---------------------------------------
\subsection{Inclusion criteria}
% ---------------------------------------

We included studies in which participants made decisions or judgments between discrete alternatives while their eye movements were recorded using eye-tracking technology. We did not include studies related to perceptual judgments, such as categorizing or discriminating visual stimuli or studies on problem solving. We excluded studies where participants were selected based on clinical diagnosis or specific socio-demographic traits e.g., visual disorders, age-related visual diseases, age restrictions such as adolescents or infants. Studies using fixed exposure time or time pressure manipulations were excluded since these manipulations can influence eye movement processes \citep{orquin2018a} and lead to substantially different results \citep{simola2019a}. Included studies used either fixation likelihood (area of interest (AOI) looked at or not), fixation count (number of fixations to AOI), total dwell time (sum of durations of all fixations to an AOI), or dwell count (number of dwells to an AOI). Eventually, 69 articles met all inclusion criteria and were included in the meta-analysis (Figure~\ref{fig:flow_diagram}). Our initial literature search retrieved 2956 articles, of which 591 remained after screening of the title and abstract. Following a more detailed evaluation of whether studies were on decision making and used eye-tracking, we identified 412 articles as potentially eligible studies. Based on detailed inspection of their full texts, 69 articles satisfied all inclusion criteria and were included in the meta-analysis. Figure~\ref{fig:flow_diagram} illustrates the PRISMA flow diagram \citep{moher2009preferred}. Many of the articles consisted of multiple experiments and some experiments operationalized more than one factor. \chg{independent}{This resulted in 122 effect size estimates, out of which 50 were effects of visual factors and 72 were effects of cognitive factors}.


\begin{figure}[H]
\includegraphics{prisma_update-crop}
\centering
\caption{The PRISMA flow diagram showing the results of the literature search.}
\label{fig:flow_diagram}
\end{figure}


% ---------------------------------------
\subsection{Data extraction and coding procedure}
% ---------------------------------------

The included studies were coded with regards to their (1) effect size, (2) sample size, (3) research domain, (4) eye-tracker model, (5) dependent variable, and (6) independent variable. As an additional measure we coded descriptive eye movement statistics for those studies reporting this i.e., the mean fixation likelihood, fixation count, total dwell time or dwell count across conditions. Although not essential to the meta-analysis, the descriptive data proved to be useful for interpreting the effect sizes. Agreement for categorical variables was assessed using Cohen's kappa and for continuous variables using intraclass correlation coefficient \citep{shrout1979a}. Overall, there was a high level of agreement: effect size, $\textrm{ICC} = 0.923$, sample size, $\textrm{ICC} = 0.996$, research domain, $\kappa = 0.731$, eye tracker model, $\kappa = 1$, dependent variable, $\kappa = 0.923$, independent variable, $\kappa = 0.934$. While revising the manuscript, effect sizes were coded again. The revised coding of effect sizes had a high agreement with the original one,  $\textrm{ICC} = 0.923$\unskip.

Coding of effect sizes is described in detail below and sample size was coded as the total number of participants in a study. The research domain was coded as preferential consumer choice, inferential consumer choice, preferential non consumer choice, inferential non consumer choice, and risky gambles. The research domain was later recoded for the analysis of choice-gaze effect in the following way: inferential consumer choice and inferential non consumer choice were recoded as inferential choice while the other three domains were coded as preferential choice. We coded the eye-tracker model as the specific name of the eye-tracking equipment used in the study, e.g. Tobii T2150 or Tobii T60, since different models from the same producer vary in measurement accuracy and precision. Information on each eye-tracker model's accuracy and precision was identified through the equipment producers' websites. We coded the dependent variable as the specific eye-tracking metric in which an effect size was reported. We coded the independent variable as visual or cognitive factors, with visual factors divided into five dimensions -- salience, surface size, left vs right position, central position, and set size -- and cognitive factors divided into three dimensions -- task instructions, preferential viewing, and choice-gaze effect. We outline these categories in detail below. 

\paragraph{Salience.} We coded studies as salience if they operationalized one or more of the known dimensions of salience such as color, edge density, contrast, or motion \citep{itti2000}. Some studies failed to indicate the direction of the salience manipulation, i.e. high vs. low levels of salience. In such cases, we contacted the original author and asked for clarification. A positive effect direction indicates that a high salience AOI receives more fixations than a low salience AOI.

\paragraph{Surface Size.} We coded studies that manipulated the relative surface size of alternatives or attribute, e.g., small vs. large alternatives or attributes \citep{lohse1997a}. Some studies manipulated the number of product facings, i.e., the number of the same product on a supermarket shelf \citep{chandon2009a}. We coded such manipulation as a surface size manipulation. A positive effect direction indicates that a large AOI receives more fixations than a small AOI.

\paragraph{Left vs right and center position.} We coded studies that manipulated the left vs right position of alternatives or attributes in horizontal arrays as left vs right position \citep{kreplin2014a}. We coded studies that manipulated the centrality of alternative or attribute position in one or two-dimensional arrays as center position \citep[experiment 1A \& 1B in][]{atalay2012a,meissner2016a}. A positive effect direction indicates for the left vs right factor that AOIs to the left receive more fixations than AOIs to the right and for centrality that AOIs in the middle receive more fixations than laterally positioned AOIs.

\paragraph{Set size.} We coded studies as set size if they manipulated the number of alternatives or attributes in a given choice task, e.g., studying the effect of a two- vs. three-alternative choice task \citep{hong2016a}. We also coded whether the set size was manipulated at the level of the alternative or the attribute. A positive effect direction indicates that each AOI receives less fixations when the set size is larger. For instance, \cite{grebitus2015} compared choices with three vs five attributes and found a lower average fixation count per AOI in the latter.

\paragraph{Task instruction.} We coded studies on task instruction if they presented participants with identical stimuli under different task instructions, e.g., testing the effect of a preferential  vs. inferential choice on eye movements \citep{orquin2019a}. We also coded whether the unit of analysis was at the level of the alternative or the attribute, i.e. whether AOIs contained alternatives or attributes. A positive effect direction indicates that an AOI receives more fixations when it is task relevant compared to not task relevant according to the task instructions.

\paragraph{Preferential viewing.} We coded studies on preferential viewing if they measured the effect of preferences on eye movements. In these studies, preference was either measured in an independent task (e.g. Becker-DeGroot-Marschak auction) or revealed through a choice in the choice task (i.e. chosen vs non-chosen alternative). We also coded whether the unit of analysis was at the level of the alternative, e.g. when participants prefer one alternative over another because it is cheaper or has a better flavor \citep{gidloef2017a}, or at the level of attributes, e.g. when price is more important than flavor \citep{meissner2016a}. A positive effect direction indicates preferred alternatives or attributes receive more fixations than less preferred ones.

\paragraph{Choice-gaze effect.} We coded studies as choice-gaze effect if they reported the difference in eye movements between the chosen alternative and all other (not chosen) alternatives. Studies that operationalized choice-gaze effect in specific time windows, e.g., the first 500 msec after stimulus onset or last 500 msec prior to choice \citep{shimojo2003a} were excluded. Based on the research domain we coded choice-gaze effect in two subfactors: preferential tasks where participants performed a preferential choice task, that is where participants were instructed to choose in accordance with their preferences \citep{schotter2010a} and inferential tasks where participants were instructed to choose in accordance with a predetermined goal, such as choosing the healthiest alternative \citep{schotter2012a}. A positive effect direction indicates that the chosen alternative receives more fixation than non-chosen alternatives.

\paragraph{Descriptive eye movement statistics.} Many studies provide descriptive statistics on the eye movement data that is used to compute effect sizes. This descriptive data is useful for understanding general tendencies in the data, e.g. the overall likelihood of decision makers fixating information or the average number of fixations to information. Additionally, descriptive statistics can be useful for translating the synthesized effect size measures into more intuitive indices such as absolute increases in fixation likelihood or fixation count. We therefore coded descriptive statistics on raw eye movements whenever studies report this. Some studies report eye movement data across conditions, e.g. fixation count across high and low salience conditions, while others report data for each condition separately, e.g. fixation count for high vs low salience conditions. We coded two types of descriptive statistics: a) overall statistics such as the average fixation likelihood or fixation count across all AOIs and conditions, and b) descriptive statistics that correspond to the extracted effect size data. An example of the latter case could be the mean fixation count in condition 1 and the mean fixation count in condition 2 which together with their pooled variances are also used to compute Cohen's $d$ and from that the effect size correlation $r$. We coded descriptive statistics for fixation likelihood, fixation count, total dwell time, and dwell count. The descriptive statistics were coded at the unit of individual AOIs rather than an entire stimulus, i.e. for a single attribute level or a single choice alternative depending on how the AOI was defined.  


% ---------------------------------------
\subsection{Construct validity of the dependent variable}
% ---------------------------------------

A possible concern in meta-analyses of eye movements is that the included studies use different eye-trackers, since data quality varies considerably across different eye-tracking equipment. Precision, which is the reliability of an eye-tracker, can vary as much as from $.005\degree$ root mean square in the best to $.5\degree$ in the poorest remote eye-trackers \citep{holmqvist2015a}. Accuracy, which is the validity of an eye-tracker, vary from around $.4\degree$ to around $2\degree$ \citep{holmqvist2015a}. With an accuracy of $2\degree$, the measured fixation, will on average fall as far as $2\degree$ away from the true fixation point. Simulations have shown that both accuracy and precision influence the capture rate, i.e., the percentage of eye movements correctly recorded within the boundaries of stimuli, which determines the degree of false positive and false negative observations \citep{orquin2019a}. The level of false positive vs. negative fixations has been shown to influence effect sizes \citep{orquin2016a}. These differences in measurement validity across eye-trackers may therefore introduce a bias in the meta-analysis of eye movements, since studies with lower accuracy and precision have lower validity, which, on average, attenuate effect sizes \citep{hunter2004a}. To inspect whether the precision and validity of eye-trackers attenuate effect sizes, and potentially correct for this, we ran a regression analysis on all included effect sizes with the absolute observed effect size correlation as the dependent variable and reported precision and accuracy of the eye-tracking equipment as the independent variables. We fitted different models using a step-up approach \citep{ryoo2011model} based on Bayesian information criterion \citep{Schwarz1978}, including models with a fixed effect for the independent variable type (salience, surface size etc.). The final model included the main effect of accuracy and a random intercept grouped by study. The second-best model also included a fixed effect for independent variable type, and the estimates of the two models were comparable. Despite analyzing across different study factors and other sources of noise, the results suggest that studies using eye-trackers with lower levels of accuracy, on average, yield lower effect sizes as predicted by the psychometric meta-analysis methods, ($\beta_0=0.429$, $\SE=0.128$, $t=3.349$, $p=0.001$, $\beta_{\textrm{accuracy}} =-0.192$, $\SE=0.13$, $t=-1.478$, $p=0.149$; Figure~\ref{fig:ET_accuracy_effectsize}). Having demonstrated that the accuracy of eye-trackers attenuates effect sizes, the next step is to correct for this phenomenon. Psychometric meta-analysis offers a method for correcting the attenuating effects of artifacts, such as the lack of validity or reliability \citep{hunter2004a}. The correction involves an artifact multiplier, $a_a$, which is a measure of the expected attenuation of the true effect size $\rho$ caused by the artifacts in study $i$. The observed study effect size $\rho_0$ is a function of the true effect size and the artifact multiplier, $\rho_0 = a_a \rho$. In the case of measurement validity, the artifact multiplier is the square root of the validity of the measurement, $a_a = \sqrt{r_{yy}}$. From this calculation, it follows that the artifact multiplier, and, hence the validity of the measurement, can be obtained as $a_a = \rho_0 / \rho$ \citep{hunter2004a}. From our model, we have estimated the observed attenuated effect size, $\rho_0$, of study $i$ as $\beta_0 + \beta_1 \textrm{accuracy}$. Given perfect accuracy, i.e. accuracy takes the value zero, the expected effect size of study $i$ is equal to the intercept, $\beta_0$, which corresponds to the expected unattenuated effect size, $\rho$. From this it follows that the artifact multiplier, $a_a$, can be computed as the ratio of the attenuated effect size proportional to the unattenuated effect size:
%
\begin{equation}
\label{eq:artifact_multiplier}
a_a = \frac{\beta_0 + \beta_1 \textrm{accuracy}}{\beta_0}
\end{equation}

For example, if a study uses an eye-tracker with an accuracy of $.50$, this yields an artifact multiplier equal to $(.569 - .382*.50)/.569 = .664$, meaning that studies with this level of accuracy will, on average, experience effect sizes that are $66.4\%$ of the true population effect size $\rho$. To compute the true average effect, $\rho$, we follow the psychometric meta-analysis method proposed by \cite{hunter2004a}. We first compute the unattenuated effect size correlation for each study, $r_i^u$, by dividing the Fisher transformed attenuated effect size with the artifact multiplier that corresponds to the level of the eye-tracker accuracy and then applying the inverse Fisher transformation, $r_i^u = \tanh(\arctanh(r_i)/a_a)$. An issue with correlation coefficients is that effect of multiplication depends on the value of the coefficient, particularly near the boundaries (-1 and 1), Fisher transformation alleviates this issue. Then, we weight each study by its sample size and its level of validity, so that studies using low accuracy eye-trackers are corrected upwards, in terms of their effect sizes and variance (Equation~\ref{eq:psychometric_rho}). A full list of eye-trackers and their accuracy and precision can be found in Table~\ref{tab:eyetracker_specifications} in Appendix~\ref{appendix}.


% ---------------------------------------
\subsection{Multiple metrics}
% ---------------------------------------

Another possible concern in meta-analyses of eye movements is that studies often rely on different eye movement metrics as their dependent variable. However, to perform a meta-analysis, we need to compare studies across a common dependent variable. The many different eye movement metrics stem from different research designs and research questions and, perhaps, also a lack of consensus about when and why to use which metrics. Many studies on visual factors report fixation likelihood while studies on cognitive factors often report fixation count, dwell count, or total dwell time (sometimes referred to as total fixation duration or total visit duration). In our analyses, we focus on fixation count since it is easier to interpret and more construct valid than the other metrics. The challenge with total dwell time is that it combines fixation count and fixation durations, but these two metrics can be correlated e.g., AOIs that have fewer fixations can have longer durations or the opposite may hold \citep{orquin2018a}. Fixation duration is influenced by factors such as the ease or difficulty of reading text or information \citep{rayner2009}, a construct that is not operationalized in the visual or cognitive factors. We therefore deem total dwell time as less construct valid in terms of the included visual and cognitive factors. Dwell count is a composite of fixation count and inter-AOI transition count i.e., how often participants move their eyes between AOIs. Inter-AOI transition count can be influenced by cognitive factors such as decision heuristics \citep{schoemann2019} or by visual factors such as the distance between pieces of information \citep{perkovic2018}. For this reason, we deem that dwell count is less construct valid for the purpose of distinguishing between the included visual and cognitive factors. Finally, fixation likelihood is a transformation of fixation count that bins all counts above zero. The binning means that in some tasks where all information is fixated there will be no effect of either visual or cognitive factors when measured in fixation likelihood, but there may still be an effect when measured in fixation count. \\

In order to inspect whether it would be meaningful to average effect sizes across different eye-tracking metrics, we reviewed the identified articles for studies that reported effect sizes in multiple metrics. Across the entire data set $43.4$\% of studies reported effect sizes in more than one metric. To investigate the strength of the relationship between the metrics, we inspected the linearity of the relationship between fixation likelihood and fixation count against other metrics by plotting all observations (Figure~\ref{fig:metric_correction}). Since the four eye movement metrics are highly correlated, we assume that the metrics are related to the same underlying construct.\\ 
While effect sizes expressed in different metrics are highly correlated, we should expect some differences between them. One mechanism that could lead to differences in effect size estimates between fixation likelihood and the remaining metrics is artificial dichotomization since fixation count, dwell count and total dwell time are treated as a binary outcome (fixated or not fixated) to produce fixation likelihood. Artificial dichotomization of a naturally continuous variable attenuates correlations with other variables \citep{hunter2004a}. We should, therefore, expect effect sizes expressed in fixation likelihood to be somewhat smaller. Correcting for artificial dichotomization requires knowledge about the true distributional split. Since none of the included studies provide information about the true distributional split of the dichotomization and since we do not have access to all data sets, we are unable to compute the artifact multiplier as proposed by \cite{hunter2004a}. Furthermore, since the eye-tracking metrics are distributed according to either zero inflated normal distribution (total dwell time) or Poisson distribution (fixation and dwell count), no such adjustments for dichotomization currently exist. Instead, we propose an empirically derived correction factor, $a_m$, to convert effect sizes expressed in one metric to another. We propose to estimate the correction factor based on our sample of studies reporting multiple metrics, by taking the ratio of the sample size weighted means expressed in the two metrics of interest:
%
\begin{equation}
\label{eq:metrics_correction}
a_m = \frac{\arctanh \left( \frac{\sum M_i^1 N_i}{\sum N_i} \right)}{\arctanh \left( \frac{\sum M_i^2 N_i}{\sum N_i} \right)}
\end{equation}
%
where $\arctanh \left( \frac{\sum M_i N_i}{\sum N_i} \right)$ is the Fisher transformed average effect size for metric $M^1$ and $M^2$, respectively weighted by sample sizes, $N$ in study $i$. The ratio is computed on the Fisher transformed effect sizes in order to meaningfully compare ratios across the whole range of correlations. For similar reasons, the correction factor is applied to Fisher transformed effect sizes which are then transformed back with the inverse Fisher transformation: $\tanh(\arctanh(r_i)*a_m)$. The method takes advantage of the fact that effect sizes from the same study expressed in different metrics control for all factors that could influence the ratio.\\    
We find that effect sizes reported in fixation likelihood are on average smaller than those reported in total dwell time and dwell count. Effect size estimates expressed in fixation likelihood are very similar to those expressed in fixation count. Table~\ref{tab:metric_correction} shows an overview of the correction factor $a_m$, that needs to be applied to convert different metrics to either fixation likelihood or fixation count. We expressed all metrics in fixation counts by applying the correction factor to each individual study effect size, but not to the study variance. When a study effect size is already reported fixation count, $a_m$ takes the value $1$.  

% ---------------------------------------
\subsection{Statistical analyses}
% ---------------------------------------
% ---------------------------------------
\subsubsection{Computation of effect sizes}
% ---------------------------------------

Effect size information was transformed into a common effect size, the Pearson’s correlation coefficient r. When multiple sources for computation of effect sizes were available, priority was given in decreasing order to other effect size measures, means and standard deviations, test statistics, beta coefficients, or p values. For studies reporting effect sizes as correlations, no further computations were performed. If a study reported p values as a threshold value, e.g., $p < .05$, we used a conservative p value equal to .05. When studies reported effect sizes for multiple AOIs, we computed the average effect size across AOIs \citep[for a similar approach, see][]{chita2016attention}. Effect sizes were extracted from the available dependent variables. Analyses were performed in R programming language with the help of several additional libraries \citep{R2020,datatable,tidyverse,metafor,irr,lme4,lmerTest,xtable,extrafont}. To reproduce our results, however, we encourage readers to use the data and code we have made publicly available at Open Source Framework at \href{https://osf.io/buk7p/?view_only=73d36c26dd794f9689c954b13c63c474}{https://osf.io}. %\citep{orquin2020osfa}.


% ---------------------------------------
\subsubsection{Weighting of effect sizes, tests of heterogeneity}
% ---------------------------------------

The effect sizes were analyzed with a psychometric meta-analysis following the approach in \cite{hunter2004a}. Individual effect sizes were first corrected using the metric correction factor, $a_m$, to yield a common dependent variable. All studies were corrected to fixation count. The psychometric meta-analysis computes the true average effect size $\rho$ based on the unattenuated correlation coefficients, $r_i^u$, weighted by sample size $n_i$, and corrected for validity by the artifact multiplier, $a_a$: 
%
\begin{equation}
\label{eq:psychometric_rho}
\rho = \frac{\sum_{i=1}^k n_i a_a^2 r_i^u}{\sum_{i=1}^k n_i a_a^2}
\end{equation}

To inspect the degree of heterogeneity in the meta-analysis, we computed the $I^2$ statistic. The $I^2$ is the proportion of variance in the observed (attenuated) effect estimates explained by artifacts and sampling error \citep{borenstein2011introduction}: 
%
\begin{equation}
\label{eq:i2_statistic}
I^2 = \frac{(T^u)^2}{(S^u)^2}
\end{equation}
%
where $(S^u)^2$ is the weighted variance of the unattenuated effect size $\rho$
%
\begin{equation}
\label{eq:Su2_var}
(S^u)^2 = \frac{\sum_{i=1}^k n_i a_a^2 (\rho_i - \hat{\rho})^2}{\sum_{i=1}^k n_i a_a^2}
\end{equation}
%
and $(T^u)^2$ is the between-studies variance component of the unattenuated effect size $\rho$
%
\begin{equation}
\label{eq:Tu2_var}
(T^u)^2 = (S^u)^2 \frac{\sum_{i=1}^k n_i a_a^2 v_i}{\sum_{i=1}^k n_i a_a^2}
\end{equation}
%
where $v_i$ is the variance of study $i$ computed as $(1 - \hat{r}^2)^2 / (n_i - 1)$ and $\hat{r}$ is the sample size weighted average effect size. Parameter estimates were obtained with restricted maximum likelihood. Because many articles report more than one effect size we computed robust variance estimates of the model coefficients based on a sandwich-type estimator with a small-sample adjustment \citep{hedges2010}.

% ---------------------------------------
\subsection{Publication bias}
% ---------------------------------------

We examined publication bias in several ways. First, we performed a precision-effect test followed by a precision-effect estimate with standard errors test \citep[PET-PEESE][]{stanley2014}. We performed PET on the Fisher z transformed effect sizes and variances. This test uses ordinary least squares to regress individual study effect sizes on study standard deviations weighted by the study precision. It is not recommended to use the test in case of large heterogeneity or on small samples, e.g. less than 10 studies \citep{vanaert2019}, and we therefore used it on the complete data set. We controlled for the artifact multiplier since the eye-tracker accuracy plays an important role in determining effect sizes and thus contributes to heterogeneity. The PEESE is used in case PET is significant, and it differs only in using the study variance instead of the standard deviation. The intercept in the PEESE is normally used as the publication bias corrected estimate. Second, because we performed the PET test on all of data (due to sensitivity to small sample sizes), we performed an additional analysis based on type of funding. It has been shown that studies which receive public grants are more likely to be published \citep{canestaro2017}. We used an inverse-variance random effects model with public grant as moderator to test for a publication bias. \chg{}{Because many articles report more than one effect size we computed robust variance estimates based on a sandwich-type estimator for both the PET-PEESE and public grants analyses. Third, as a sensitivity analysis at the level of each independent variable we used the Top10 most precise method \citep{stanley2010}. This method consists of discarding the 90\% most imprecise studies and performing the meta-analysis on the remaining 10\%. While seemingly paradoxical, simulations have shown that in the presence of publication bias the Top10 method greatly reduces publication selection bias in the synthesized effect size estimates. Since the same meta-analysis model is used on the complete data and the 10\% most precise data, the Top10 method can be applied to any data structure or statistical model. Other methods such as trim and fill \citep{duval2000trim} or p-uniform \citep{vanassen2015} are, for instance, not consistent with our psychometric meta-analysis with robust variance estimation. We computed precision as the inverse of the artefact adjusted study variance $1/v_i$ where the variance $v_i$ of study $i$ was defined as:
%
\begin{equation}
v_i = \frac{(1 - \hat{r}^2)^2 / (n_i - 1)}{a_a^2}
\end{equation}
%
where $\hat{r}$ is the sample size weighted average effect size, $n_i$ is the sample size of study $i$, and $a_a$ is the artefact multiplier correcting for study validity. Some independent variable groups contained very few observations, and for the sake of comparability we therefore included the two most precise studies from each group.} In the \textit{Appendix} we present funnel plots of the relationship between effect sizes and standard errors for each independent variable group.


% ---------------------------------------
\subsubsection{Descriptive eye movement averages and relation to effect sizes}
% ---------------------------------------
%
To compute average eye movement measures, we first computed the mean across conditions for studies reporting descriptive eye movement data within conditions. We then appended this data with that of studies directly reporting descriptive data across conditions. Average eye movement measures were then computed based on the appended data set. To provide intuitions about the synthesized effect sizes, we examined the relationship between individual study effect size correlations and the corresponding descriptive eye movement data for those studies reporting within conditions. For fixation likelihood (FL) descriptive statistics, we logit transformed it, $logit(FL) = log(FL/1 - FL)$, and then computed the logit difference, $logitD$, between conditions (a and b) for each study, $logitD = logit(FL_{a}) - logit(FL_{b})$. Next we regressed the resulting $logitD$ variable on effect size correlations expressed in fixation likelihood using a linear mixed model with a random intercept, grouped by article to account for correlated errors. Using the model coefficients we computed for each independent variable the equivalent logit difference, $logitD_{IV} = \beta_{0} + \beta_{1}\rho_{IV}$. We then computed the inverse logit of $logitD_{IV}$ and the logit of the average fixation likelihood, $logit(\overline{FL})$, to get the expected increase in fixation likelihood for a study with an average fixation likelihood, $FL increase = 1 / 1 + e^{LogitD + logit(\overline{FL})}$. We performed a similar computation to express effect sizes in fixation count and total dwell time, with the only exception that instead of a logit and inverse logit transformations we used logarithmic and exponential transformations. We did not perform any transformations on dwell count since there was insufficient data to produce reliable estimates.
% -------------------------------------------------------
% Results
% -------------------------------------------------------

\section{Results}

Meta-analyses of eye movements are relatively rare, potentially because of methodological challenges in combining effect sizes from different eye-tracking studies. Two main challenges are how to handle measurement validity across eye-tracker types and how to compare different eye movement dependent variables. To handle these issues, we developed correction procedures integrated in a psychometric meta-analysis \citep{hunter2004a}, which allow us to quantify the interference of measurement validity or multiple metrics. The measurement validity issue stems from differences in the accuracy and precision of eye-tracking equipment \citep{holmqvist2015a}, which can affect the data quality and bias effect sizes \citep{orquin2016a}. We developed a correction method that relies on an empirical estimate of the relationship between eye-tracker characteristics and observed effect sizes (see \textit{Method}; Figure~\ref{fig:ET_accuracy_effectsize}; Table~\ref{tab:eyetracker_specifications}). Most metrics are based on fixations -- defined as maintaining the gaze at a single location or area of interest (AOI), such as fixation count, fixation likelihood, total dwell time, and so on. This leads to a potential issue with comparing effect sizes reported with different dependent variables. We developed a correction method that makes the dependent variables comparable, where we empirically estimate correction factors based on a subset of studies in our sample that report multiple dependent variables (see \textit{Method}; Figure~\ref{fig:metric_correction}; Table~\ref{tab:metric_correction}). This method allowed us to transform all effect sizes to a single metric; we decided for fixation count, which was used in all meta-analyses. 

To understand the context in which the studies were conducted, we coded participant sample information -- age, gender, ethnicity, and country -- across visual and cognitive factors (Table~\ref{tab:sampleTable} in the Appendix). Generally, participant samples include a roughly even split of men and women in many cases with a mean age above 30. The studies are collected in a wide range of countries, but are with few exceptions restricted to white participants. 

In what follows, we first analyze the group of visual factors and then the group of cognitive factors. We perform a meta-analysis on each individual factor  separately. We next perform a small moderator analysis and finish with an analysis of publication bias in all the meta-analyses.


\subsection{Almost all visual factors have substantial effect sizes}

We focused on four major groups of visual factors -- salience, position, surface size, and set size (see \textit{Method} for coding procedure). The summary effects of the visual factors on attention during decision making show that the factors range from small to medium effect sizes (see Table~\ref{tab:main_results} and Figure~\ref{fig:forest_plots_visual}). Since the publication bias analyses suggest the presence of bias (see \textit{Publication bias} section below), we conducted a sensitivity analysis that indicates the possible extent of bias in individual factors. The Top10 sensitivity analysis (see \textit{Method} for details) suggests large upward adjustment for three factors -- salience, surface size, and center position -- resulting in medium to large effect sizes (Table~\ref{tab:main_results}; results in parentheses). For set size the Top10 analysis suggests no change in effect size, while for left vs. right position it suggests large downward correction to a near null effect. While the left vs. right position does not seem to have an effect on fixation count, we cannot exclude effects on temporal dynamics. For instance, decision makers may be more likely to begin with the left option and then move right \citep{fiedler2012}, which could lead to primacy effects on choice. Overall, the Top10 analysis suggests that there is some publication bias in the visual factors. Inspecting the effect sizes, it is interesting that salience, which so far has taken the center stage in vision science, has smaller summary effect than some of the other visual factors. The largest effects are those of center position, $r=.43$, and surface size, $r=.35$, which are on par with cognitive factors, which range between $r = .35$ and $r = .59$. If considering the Top10 analysis results, then visual factors are substantially larger than cognitive factors. While it is possible to eliminate the effects of, for instance, center position and surface size in laboratory studies, this is not true for natural environments. In natural environments it is reasonable to expect that multiple, and most likely all, visual factors influence eye movements at the same time. Not only do the effects of the largest visual factors exceed the largest cognitive factors when comparing one by one, but considering their joint effect we believe it is reasonable to conclude that visual factors play a larger role than cognitive factors in determining visual attention in decision making. 


% latex table generated in R 3.5.2 by xtable 1.8-4 package
% Fri Nov 20 13:40:06 2020
\begin{table}[ht]
\centering
\caption{Main results of the meta-analysis, divided into visual and cognitive factor groups, and individual factors within them. The most important values are the corrected effect size estimate, $\rho$, and the associated heterogeneity, $I^2$. Results of trim and fill analysis are in the parentesis.} 
\label{tab:main_results}
\begingroup\small
\begin{tabular}{lccccccccc}
  \hline
Group & $k$ & $N$ & $\rho$ & SE & $Z$ & $p$ & $\textrm{CI}_{95}$ LL & $\textrm{CI}_{95}$ UL & $I^2$ \\ 
  \hline
\textbf{Visual factors} &  &  &  &  &  &  &  &  &  \\ 
  Salience & 12 & 1251 & 0.13 & 0.04 & 3.46 & <0.001 & 0.06 & 0.21 & 0 \\ 
   & (4) &  & (0.18) & (0.03) & (5.47) & (<0.001) & (0.12) & (0.25) &  \\ 
  Surface size & 9 & 1425 & 0.35 & 0.07 & 5.02 & <0.001 & 0.21 & 0.49 & 69.57 \\ 
   & (0) &  & (0.35) & (0.07) & (5.02) & (<0.001) & (0.21) & (0.49) &  \\ 
  Left vs right position & 5 & 552 & 0.24 & 0.13 & 1.84 & 0.065 & -0.02 & 0.5 & 61.17 \\ 
   & (3) &  & (0.07) & (0.16) & (0.46) & (0.644) & (-0.24) & (0.39) &  \\ 
  Center position & 11 & 1134 & 0.43 & 0.08 & 5.13 & <0.001 & 0.26 & 0.59 & 69.13 \\ 
   & (0) &  & (0.43) & (0.08) & (5.13) & (<0.001) & (0.26) & (0.59) &  \\ 
  Set size & 13 & 1007 & 0.24 & 0.07 & 3.7 & <0.001 & 0.12 & 0.37 & 58.28 \\ 
   & (2) &  & (0.2) & (0.07) & (2.8) & (0.005) & (0.06) & (0.34) &  \\ 
  \textbf{Cognitive factors} &  &  &  &  &  &  &  &  &  \\ 
  Task instructions & 28 & 2060 & 0.35 & 0.05 & 6.51 & <0.001 & 0.25 & 0.46 & 52.72 \\ 
   & (4) &  & (0.32) & (0.06) & (5.42) & (<0.001) & (0.21) & (0.44) &  \\ 
  Preferential viewing & 24 & 2633 & 0.36 & 0.07 & 4.99 & <0.001 & 0.22 & 0.5 & 80.64 \\ 
   & (6) &  & (0.26) & (0.08) & (3.13) & (0.002) & (0.1) & (0.43) &  \\ 
  Choice bias & 18 & 835 & 0.59 & 0.08 & 7.32 & <0.001 & 0.43 & 0.75 & 63.76 \\ 
   & (7) &  & (0.41) & (0.09) & (4.38) & (<0.001) & (0.22) & (0.59) &  \\ 
   \hline 
 \multicolumn{10}{p{0.95\textwidth}}{\scriptsize{\textit{Note.} $k$ = number of studies (for trim and fill analysis number of imputed studies); $N$ = number of participants; $\rho$ = unattenuated effect size estimate, SE = standard error of estimate; $Z$ = Z statistic; $p$ = significance level; $\textrm{CI}_{95}$ LL = lower limit of the 95\% confidence interval; $\textrm{CI}_{95}$ UL = upper limit of the 95\% confidence interval, $I^2$ = within-group heterogeneity.}} 
\end{tabular}
\endgroup
\end{table}



\begin{figure}[!h]
\includegraphics{forest_plots_visual}
\centering
\caption{Effect sizes of the visual factors are moderate, except for salience and left vs. right position, which have small effect sizes, if any. Forest plots show the unattenuated effect size correlations for each study in a group, as well as average effect across the group. Forest plot in (A) shows the effect sizes for salience factor, in (B) for center position, in (C) for left vs. right position, in  (D) for surface size, and in (E) for set size factor. Error bars represent the 95\% confidence interval around the mean.}
\label{fig:forest_plots_visual}
\end{figure}


\subsection{Cognitive factors have effect sizes similar in magnitude to visual factors}

Previous research has identified a wide range of cognitive factors that influence attention, such as goals, task instructions, and preferences \citep[for a review see][]{orquin2013a}. Here, we divided cognitive control factors into three groups: task instruction, preferential viewing, and choice-gaze effect. In studies on task instructions, participants receive instructions concerning a specific decision goal, and with that, what is it relevant to gaze at. For instance, the participants may be instructed on the validity of stimulus attributes \citep{krefeld-schwalb2019a}, or infer the level of validity themselves \citep{bialkova2014a}. In preferential viewing studies, the relevance should be equal to the subjective preferences. For example, some alternatives have higher subjective values than others \citep{kim2012a}. Because of this qualitative difference between the two domains, we treated studies on task instructions and preferential viewing separately. 
Inspecting the effect sizes reveals that the summary effects in the two types of studies are moderate and similar in magnitude (see Table~\ref{tab:main_results} and Figure~\ref{fig:forest_plots_cognitive}). Using a Wald test, we find that effect sizes of task instructions and preferential viewing are unlikely to differ, $z=-0.033$, $p=0.399$. 
The test suggests that it makes no difference to eye movements whether the relevance of information is defined according to an externally specified goal or according to preferences. Note, however, that the two effect sizes differ more when comparing the effect sizes from the Top10 analysis, which suggests downward adjustments, with larger adjustment for task instructions (Table~\ref{tab:main_results}; results in parentheses).

Choice-gaze effect refers to an effect in attention whereby decision makers spend more time gazing at the eventually chosen alternative. This effect, originally introduced by Shimojo and colleagues \citep{shimojo2003a} as a ``gaze-cascade'' effect, is well-established in the literature, prompting us to study it as a separate factor. This factor consists of studies reporting the difference in eye movements between the chosen alternative and all other (not chosen) alternatives. We find that choice-gaze effect has a large effect on eye movements, $r=.59$. However, the Top10 analysis suggests a large downward adjustment, to a small effect, $r=.26$.


\begin{figure}[!h]
\includegraphics{forest_plots_cognitive}
\centering
\caption{Effect sizes of the three cognitive factors are moderate to large. Forest plots show the unattenuated effect size correlations for each study in a group, as well as average effect across the group. Forest plot in (A) shows the effect sizes for task instructions factor, in (B) for preferential viewing, and in (C) for the choice-gaze effect factor. Error bars represent the 95\% confidence interval around the mean.}
\label{fig:forest_plots_cognitive}
\end{figure}


\subsection{Alternative vs. attribute moderator is significant only for the set size factor}

Alternatives that participants in judgment and decision making studies choose between can often be decomposed into constituent elements, commonly called attributes, cues, or features \citep{payne1988,tversky1972elimination,stojic2020s,gigerenzer1996reasoning,schulz2018putting,hogarth2007heuristic}. For example, in classical lottery tasks \citep{tversky1979}, the probabilities and values of an alternative can be viewed as attributes. Or, in multi-cue judgment tasks, alternatives are more explicitly composed of cues -- in the German city size task, for example, these would be university, major football team, main city, and so on \citep{gigerenzer1996reasoning}. This has consequences for both modelling of decision processes and units of analysis. Consequently, some studies in our sample focused on attention effects at either alternative or attribute level, or both. This was in particular the case for studies involving set size, task instructions, and preferential viewing factors. Since the alternative vs. attribute dimension might be an important moderator in these groups, we decomposed them further with regards to the effect of alternatives vs. attributes (Table~\ref{tab:mod_results} and Figure~\ref{fig:forest_plots_altatt}). Moderator analyses show support for the alternative vs. attribute moderator across set size, $t=-6.604$, $p=0.008$\unskip, but no support for preferential viewing, $t=-1.776$, $p=0.199$\unskip, or for task instructions, $Q_M(1)=1.947$, $p=0.163$\unskip. It is worth noting that effect sizes are consistently larger when operationalized at the level of alternatives compared to attributes (Table~\ref{tab:mod_results} and Figure~\ref{fig:forest_plots_altatt}). 

We also performed a moderator analysis for the choice-gaze effect factor, to assess whether the effect is driven by preferential viewing as proposed by \cite{shimojo2003a}. We compare studies with preferential vs. inferential choice tasks and find no support for moderation by decision type, $Q_M(1)=0.057$, $p=0.811$\unskip, and therefore only report results for the main group. 


% -------------------------------------------------------
% Publication bias
% -------------------------------------------------------

\subsection{Publication bias exists, but mainly affects cognitive factors}

We assessed potential publication bias by first performing a precision-effect test \citep[PET][]{stanley2014}, which regresses individual study effect sizes on study standard deviations weighted by the study precision, controlling for the eye-tracker accuracy in addition (see \textit{Method} for details). It is not recommended to use the test in case of small samples (for example, less than 10 studies \citep{vanaert2019}), and we therefore used it on the complete data set. Because many articles report more than one effect size we computed robust variance estimates of all model coefficients based on a sandwich-type estimator. While the PET is not significant, $\beta=0.17$, $SE=0.1$, $p=0.28$\unskip, Egger's test shows a significant effect of standard deviation of the effect size, $\beta=2.16$, $SE=0.62$, $p=0.01$\unskip, suggesting the presence of publication bias (see \textit{Appendix} Table \ref{tab:PET-PEESE} for full regression results). Given the significant Egger's test, we then performed the precision-effect estimate with standard errors test (PEESE). The PEESE differs only in using the study variance instead of the standard deviation. The intercept in the PEESE is normally used as the publication bias corrected estimate, $\beta=0.23$, $SE=0.1$, $p=0.17$ (see \textit{Appendix} Table \ref{tab:PET-PEESE} for full regression results). The bias corrected estimate in the PEESE test is, in our case, not very informative by itself since it groups all independent variables into one estimate. However, by comparing the PEESE intercept to that obtained from an intercept only regression, % latex table generated in R 3.5.2 by xtable 1.8-4 package
% Wed Dec  2 10:37:38 2020
\begin{table}[ht]
\centering
\caption{Fixed effects analysis of complete data} 
\label{tab:FE}
\begingroup\small
\begin{tabular}{llccc}
  \hline
Parameter & Estimate & SE & $t$ & $p$ \\ 
  \hline
Intercept & 0.29 & 0.03 & 10.82 & 0.00 \\ 
   \hline
\end{tabular}
\endgroup
\end{table}
\unskip, we can get an impression of the degree of effect size overestimation due to publication bias. This comparison suggests that publication bias leads to an overestimation factor of \input{tables/peeseFactor}\unskip.

We performed an additional analysis of publication bias because of the above-mentioned limitations of the PET-PEESE test. It has been shown that studies which receive public grants are more likely to be published \citep{canestaro2017}, and we therefore expected that acknowledging a public grant in the article may be associated with smaller effect sizes. We used a random effects model with robust variance estimation and public grant as moderator (see \textit{Method}). We find that although public grants are associated with smaller effect sizes, the moderator does not have a significant effect, $Q_M(1)=0.12$, $p=0.73$\unskip. Comparing public vs. non-public funded research we derived an overestimation factor of \input{tables/publicFactor}\unskip, which is nevertheless higher than the estimate derived from the PET-PEESE analysis.

The PET-PEESE and public grant analyses are based on very different premises, but they both indicate the presence of publication bias. We therefore proceeded with a Top10 analysis, an additional sensitivity analysis at the level of each independent variable (see \textit{Method}). The Top10 analysis resulted in an upward adjustment of the average effect size for several visual factors, downward adjustment for all cognitive factors and the visual factor left vs. right position. The corrected effect sizes in Table~\ref{tab:main_results} (in parentheses) provide a more conservative estimate of the true population effects, but are also subject to some uncertainty because of a relatively small number of studies in the visual factor groups. Comparing the Top10 with the uncorrected estimates suggests an overestimation factor of \input{tables/trimFactor}\unskip, which is close to the overestimation factors from the PET-PEESE and the public grant analysis.

The three analyses suggest an overestimation factor somewhere in the range of \input{tables/peeseFactor} to \input{tables/trimFactor}\unskip. While any factor above 1 is undesirable, it is worth noting that it is far below the factor of 2.59 suggested by \cite{kvarven2020} who compared 15 pairs of meta-analysis estimates with estimates from many-lab replication studies. In addition to these analyses, we plotted the Fisher transformed correlation coefficients of each study by its respective standard error (so-called funnel plots; Figure~\ref{fig:funnel_plots} for main results, and Figure~\ref{fig:funnel_plots_altatt} for moderator analyses). The symmetry of the funnel plots provides a qualitative picture of publication bias since we expect that studies with smaller sample sizes and hence higher standard errors yield more variable effect sizes, the smallest of which are less likely to be published, leading to an asymmetric funnel plot. Funnel plots are generally rather difficult to interpret, and in our case it is almost impossible to draw any conclusion for visual factors, given the smaller number of studies on which they are based.


% -------------------------------------------------------
% Descriptive EM analysis
% -------------------------------------------------------


\subsection{Projecting effect sizes to original measurement units reveals substantial effects in absolute sense}

To better understand the results of the meta-analysis we extract for each effect size the corresponding descriptive eye movement data whenever the included study reports this information (see \textit{Method}). For example, a study would report a fixation count (or likelihood or dwell time) for high vs. low salience conditions. We extracted the descriptive eye movement data at the unit of a single AOI, for example, corresponding to the fixation count for a single attribute level or for a single alternative if the AOIs are defined at this level. A total of \protect{\input{tables/authorEMcount}\unskip} articles report descriptive eye movement data, resulting in \protect{\input{tables/EMcount}\unskip} corresponding pairs of effect sizes and descriptive eye movement data, with some studies reporting results for more than one dependent variable. In total, there are \protect{\input{tables/flEMcount}\unskip} studies reporting fixation likelihood, \protect{\input{tables/fcEMcount}\unskip} reporting fixation count, and \protect{\input{tables/tdtEMcount}\unskip} reporting total dwell time.


\begin{figure}[!h]
\includegraphics{figs/EMtoES.pdf}
\centering
\caption{Descriptive eye movement data provides insight into the attention behavior of an average participant and how effect sizes translate into the original measures on which they were based. Distribution plots illustrate the fixation likelihood (A), the average fixation count (B), and the average total dwell time (C) across studies. We illustrate the linear relationship between the effect size correlation and the descriptive eye movement data by logit transforming differences in fixation likelihood (D), log transforming fixation count (E), and log transforming total dwell time (F). Lines are based on intercepts and slopes from linear mixed model fits (see main text and \textit{Method} for details).}
\label{fig:em_figure}
\end{figure}


From averaged eye movement measures we see that the average for fixation likelihood is $M=0.614$, $SD=0.233$\unskip, for fixation count $M=7.806$, $SD=13.547$\unskip, and total dwell time $M=2253.505$, $SD=4258.481$\unskip. From the distribution plots it is clear that there is a lot of variance in fixation likelihood across studies (Figure~\ref{fig:em_figure}). In some studies participants fixate nearly all AOIs, but there is also a large number of studies in which participants fixate half or less of the AOIs. The main use of the descriptive eye movement data, however, was to provide intuitions about the synthesized effect sizes. To this end, we transformed the synthesized effect size for each independent variable into its corresponding effect on fixation likelihood, fixation count, and total dwell time (see \textit{Method} for details). We achieved this transformation by regressing descriptive measures (appropriately transformed) on effect size correlations, using a linear mixed model with a random intercept, grouped by article to account for correlated errors. For fixation likelihood (logit difference between conditions), the model intercept is not significantly different from zero, whereas the slope is, $\beta_{0}=0.17$, $SE=0.13$, $p=0.21$, $\beta_{1}=2.67$, $SE=0.42$, $p< 0.001$\unskip. Figure \ref{fig:em_figure} panel (D) illustrates the relationship between the transformed statistic and effect size correlations, with an increasing variance as effect sizes become larger. For fixation count (log difference between conditions), the model shows a similar pattern where the intercept is not significantly different from zero, while the slope is, $\beta_{0}=0.11$, $SE=0.09$, $p=0.27$, $\beta_{1}=1.71$, $SE=0.3$, $p< 0.001$\unskip, as illustrated in Figure~\ref{fig:em_figure} panel (E). The model for total dwell time (log difference between conditions) reveals the same pattern, $\beta_{0}=0.11$, $SE=0.09$, $p=0.23$, $\beta_{1}=1.52$, $SE=0.39$, $p< 0.001$\unskip, as illustrated in Figure~\ref{fig:em_figure} panel (F). Overall, all three measures strongly correlate with the (transformed) effect sizes, giving us confidence for converting effect sizes into original measures using the fitted models. More specifically, for each independent variable we computed the expected increase in descriptive eye movement measures based on the effect size, for a study with an average descriptive measure (see \textit{Method} for details). Combining the estimates from these operations we could finally compare the effect sizes for each independent variable in terms of the equivalent effect on fixation likelihood, fixation count, and total dwell time for an average study (see Table~\ref{tab:em_results}). The table reveals surprisingly large effects on eye movements when expressing effect size correlations in raw eye movement metrics. Even the smallest effect observed for salience, $\rho = .13$, leads to an increase in fixation likelihood from an average of $61\%$ to $76\%$. Using a central location increases fixation likelihood to $91\%$. For fixation count the numbers are equally surprising. From an average fixation count of $7.8$, increasing salience can add three more fixations to a piece of information, while using a central location can add nearly ten more fixations. For the purpose of policy interventions aiming to capture decision maker attention, these numbers are beyond all our expectations.

% latex table generated in R 3.6.3 by xtable 1.8-4 package
% Fri Dec 18 00:44:23 2020
\begin{table}[ht]
\centering
\caption{Main effects expressed as absolute changes in the fixation likelihood, fixation count, and total dwell time. The lower bounds correspond to an average study in the data set while the upper bounds correspond to the absolute change in eye movement metrics as a consequence of the visual and cognitive factors.} 
\label{tab:em_results}
\begingroup\small
\begin{tabular}{p{3.7cm}p{1.2cm}p{1.3cm}p{1.3cm}p{1.3cm}p{1.3cm}p{1.6cm}p{1.6cm}}
  \hline
Group & $\rho$ & $\textrm{FL}_{LL}$ & $\textrm{FL}_{UL}$ & $\textrm{FC}_{LL}$ & $\textrm{FC}_{UL}$ & $\textrm{TDT}_{LL}$ & $\textrm{TDT}_{UL}$ \\ 
  \hline
Salience & 0.13 & 0.61 & 0.76 & 7.77 & 10.80 & 2252.39 & 3217.02 \\ 
  Surface size & 0.35 & 0.61 & 0.87 & 7.77 & 15.73 & 2252.39 & 4570.51 \\ 
  Left vs right position & 0.24 & 0.61 & 0.82 & 7.77 & 13.03 & 2252.39 & 3834.50 \\ 
  Center position & 0.43 & 0.61 & 0.90 & 7.77 & 18.03 & 2252.39 & 5193.06 \\ 
  Set size & 0.24 & 0.61 & 0.82 & 7.77 & 13.03 & 2252.39 & 3834.50 \\ 
  Task instructions & 0.35 & 0.61 & 0.87 & 7.77 & 15.73 & 2252.39 & 4570.51 \\ 
  Preferential viewing & 0.36 & 0.61 & 0.88 & 7.77 & 16.00 & 2252.39 & 4644.05 \\ 
  Choice-gaze effect & 0.59 & 0.61 & 0.94 & 7.77 & 23.69 & 2252.39 & 6704.09 \\ 
   \hline 
 \multicolumn{8}{p{0.95\textwidth}}{\scriptsize{\textit{Note.} $\textrm{FL}_{LL}$ = lower limit of fixation likelihood, $\textrm{FL}_{UL}$ = upper limit of fixation likelihood, $\textrm{FC}_{LL}$ = lower limit of fixation count, $\textrm{FC}_{UL}$ = upper limit of fixation count, $\textrm{TDT}_{LL}$ = lower limit of total dwell time, $\textrm{TDT}_{UL}$ = upper limit of total dwell time.}} 
\end{tabular}
\endgroup
\end{table}


% -------------------------------------------------------
% Discussion
% -------------------------------------------------------

\section{Discussion}

%%% 1. Brief reminder what the study is about
For the better part of our daily lives, we attend to and gather information using our eyes and consequently many of the decisions we make, small or large, depend on visual attention. In this article, we attempt to answer to what extent the visual environment guides our attention during decision making. To this end, we meta-analyze empirical studies on eye movements in decision making. We distinguish between visual environment factors such as salience, surface size, set size, and position, and compare them to cognitive factors such as preferential viewing, task instructions and choice-gaze effect. \chg{}{We identify 122 effect sizes from 69 articles} and perform a psychometric meta-analysis to control for methodological issues that arise when meta-analysing eye-movement studies.\\ 

% main findings - importance of visual factors 
% att odds with vast majority of current theories
Except for salience and left vs right position, the results show that visual factors have medium effect sizes. In comparison, effect sizes of the three cognitive factors are slightly larger, choice-gaze effect in particular. \chg{}{However, when controlling for publication bias, the visual factors are become larger than the cognitive factors.} In laboratory environments, it is possible, and often desirable, to control for visual factors, but in natural environments where no such control or counterbalancing takes place, all visual factors could influence eye movements simultaneously \citep{gidloef2017a, orquin2019a}. \chg{other-factors}{Furthermore, there are potentially other visual factors not covered in our study that influence eye movements. For instance, light and shade, texture, or occlusion by objects  \citep{geisler2008}, gestalt principles such as the laws of proximity, similarity, closure etc. \citep{wagemans2012}, or overall image properties such as feature or design complexity \citep{pieters2010a} or visual clutter \citep{rosenholtz2007a}. To the best of our knowledge none of these factors have been studied in the context of decision making.} Thus, visual factors might be major drivers of attention in real world decision making, well aligned with previous suggestions that 2/3 of variance in eye movements is due to visual factors \citep{vanderlans2008}. These findings are clearly at odds with most decision making models that assume equal attention to all stimuli \citep{tversky1979,payne1988, simon1956a}, but also with models that assume no role of cognitive factors in guiding attention in decision making \citep{busemeyer1992, krajbich2010a} or no role of visual factors in guiding attention \citep{callaway2019a, gloeckner2011a, gluth2018, gluth2020}.\\ 

% integrating visual and cognitive factors in models of attention and
% decision making
Our findings will hopefully reinvigorate the line of research integrating visual and cognitive factors in driving attention in decision making. Important first steps have been taken by \cite{chen2013}, \cite{navalpakkam2010}, and \cite{towal2013a}, who developed models integrating the role of salience in decision making. Their sequential sampling based models suggest that salience may influence the onset of drift or perhaps the amount of drift. This research left us with some important questions unanswered and new research should tackle these first. For example, we still do not know whether salience consistently biases attention in decision situations, or if the effect is limited to decisions under time pressure as in the before mentioned studies? If salience mainly influences attention immediately after stimulus onset \citep{theeuwes2010, orquin2015a}, the effect of salience on attention and choice may diminish as the decision time extends or it may have no bearing on the effect if salience influences the onset of drift as suggested by \cite{chen2013}. While there are still many unanswered questions about the mechanisms underlying the interactions between salience and decision processes, hardly any have been addressed concerning the other visual factors. Our findings are silent on the mechanisms and a pressing next step is to integrate multiple visual factors in decision making models to improve our understanding how exactly they jointly affect attention and possibly choices. A good starting point is to investigate visual factors with larger effect sizes identified in the present study -- surface size, center position, and set size -- alongside salience that has been studied previously. \chg{r427}{A promising modeling approach to account for visual and cognitive factors in decision making is to use similar model architecture for both attention and choice processes. Sequential sampling models are one option, as used in \cite{towal2013a}. Another option are interactive activation models that have been used for modelling both perceptual processes \citep{mcclelland1981} and choice processes, as in the parallel constraint satisfaction model \citep{gloeckner2011a}.} \chg{r426}{This model took an important step towards integrating the perceptual and choice domain and has, for instance, led to the discovery of the attraction search effect i.e. that decision makers are more likely to search attributes of already favored alternatives \citep{jekel2018}.} \\   

% what is a visual factor? limits of model-free definitions
% underscores the need for further modelling development
For the set size factor we observed the effect was moderated by alternative vs attribute, which reveals some limits of model-free classifications into visual and cognitive factors. We find a larger effect of set size by alternatives than set size by attributes, which implies that decision makers are more likely to ignore information when the set size increases in number of alternatives rather than in number of attributes. This finding suggests that, even though we have presented set size as a visual factor, it may influence the decision process as a cognitive factor, by moderating the search stopping point. Prior studies on multi-alternative decision making \cite{reutskaja2011, stuttgen2012, thomas2020} suggest that decision makers may rely on satisficing or a hybrid of satisficing for determining when to stop a search process. However, neither satisficing nor the proposed hybrid satisficing models can account for our findings on set size effects since these models assume that stopping is independent of the set size. This finding underscores the need for an integrative treatment of visual and cognitive factors in models of attention and decision making. This is the best way forward to improve our understanding of these findings and underlying mechanisms.\\

% external instructions and preferential viewing have the same 
% effects, further studies needed to examine whether attention
% process is really the same
Regarding cognitive factors, we decided to analyze studies on task instructions and preferential viewing separately since there is a clear qualitative difference between the two domains. In studies on task instructions, participants receive instructions concerning a specific decision goal, whereas, in preferential viewing studies, participants decide based on subjective preferences. The inspection of the effect sizes reveals that the main effect in the two types of studies are practically indistinguishable. This result suggests that it makes no difference to eye movements whether the relevance of information is defined according to an externally specified goal or according to subjective preferences. Breaking down both groups by alternatives and attribute moderators reveal further similarities. Although moderator analyses show a weak effect for preferential viewing and no effect for task instructions, in both cases there is a larger effect at the alternative level. An important caveat is that while effect sizes might be similar, the attention patterns behind them need not be. In other words, while both influence fixation count to a similar degree the order or timing of fixations could differ. Further research is necessary to determine whether preferential choice and choice according to external goals entail the same attention process as implied by, for instance, sequential sampling models \citep{forstmann2016}.\\ 

% choice-gaze effect, the biggest effect and still unresolved
Choice-gaze effect has the largest effect on eye movements in our study. The choice-gaze effect is similar for preferential and inferential studies, suggesting that the effect is not driven by preferential viewing. Even in tasks where participants are instructed to choose their least preferred alternative, they have more fixations to the chosen alternative. There are several theories predicting choice-gaze effect. One theory is that choice-gaze effect arises because of the gaze cascade phenomenon \citep{shimojo2003a}, but our findings suggest this cannot be the case since both preferential and inferential choices result in choice-gaze effect. Alternatively, choice-gaze effect could result from an evidence accumulation process as proposed in the attentional Drift Diffusion Model \citep{krajbich2010a}. The aDDM implies that the last fixation is often to the chosen alternative which could increase the fixation time or count for that alternative. However, effect size of the choice-gaze effect is substantial and most likely results from more than a single extra fixation to the chosen alternative. The aDDM is therefore not a good explanation for the choice-gaze effect phenomenon. Another possibility is that choice-gaze effect is the result of a process in which decision makers prioritize attention towards high-value alternatives as they learn about the values of the choice alternatives. There are several competing models that all imply a gradual orientation of attention towards high value alternatives \citep{callaway2019a, gloeckner2011a, manohar2013} and simulation studies may shed light on their ability to fully account for the choice-gaze effect phenomenon. A final possible explanation is that choice-gaze effect is the consequence of preparations for a motor response towards the chosen alternative \citep{hayhoe2014a}. This mechanism could furthermore contribute to choice-gaze effect along with other mechanisms such as the attention prioritizing process. The specific mechanism behind choice-gaze effect remains unclear; but considering how large the effect is, and the number of models that imply this effect, we believe that a better, and eventually full understanding of the effect will help advance decision research.\\ 

% Impact on broad range of disciplines
Our findings have implications for several scientific disciplines. Disciplines such as cognitive psychology, behavioral economics, and marketing are well represented in the set of included studies. For these disciplines, our findings provide a useful framework for developing successful behavioral interventions or marketing communication based on visual factors \citep{muenscher2016a, orquinwedel2020}. Our findings also point to the possibility of measuring individual preferences in real time through eye movements -- a technique that is becoming increasingly relevant as many everyday devices have built-in cameras that can serve as eye-trackers \citep{bulling2019a}. It is currently possible to perform low-resolution eye-tracking at home using a computer and web camera and preferential viewing could, for instance, serve as an implicit measure of preferences for a large sample of consumers. For vision science, our findings are particularly relevant being possibly the first meta-analysis to compare the effect of visual and cognitive factors on eye movements and may help refine gaze models of search \citep{vanderlans2008} and natural tasks \citep{hayhoe2005}. Other disciplines may want to take stock of these findings and to evaluate the generalizability of the findings to their respective discipline. Given the high degree of variance in methods and stimuli, we expect that our results generalize well to disciplines such as learning and education research, problem solving, or human-computer interaction. However, disciplines studying eye movements in natural environments, e.g., driving, aviation, or other natural tasks, should be cautious when applying our findings since the vast majority of the included effect sizes were from laboratory-based studies.\\ 

% \subsection{Methodological contributions}
Only a few meta-analyses have been published on eye movements and no guidelines exist on how to handle eye-tracking-specific issues in meta-analyses. To perform our analysis, we have developed procedures for how to handle issues related to multiple metrics and eye-tracker validity. The procedure for handling eye-tracker validity showed that eye-trackers with poorer accuracy, in general, lead to lower effect sizes. In our data, the difference in validity as indicated by the artefact multiplier ranged from .36 to .85 between the best and worst eye-trackers (see Table~\ref{tab:eyetracker_specifications}). This result is a substantial difference. Accounting for eye-tracker validity improved the precision of the synthesized effect sizes. This finding is an important methodological contribution which demonstrates the relevance of ensuring high-quality eye-tracker data. Eye movement related dependent variables come in multiple metrics such as fixation count, fixation likelihood, or dwell count. We showed that these metrics yield similar effect sizes and developed a method for converting effect sizes expressed in one metric into another. This method will allow future eye movement meta-analyses to overcome this important practical obstacle. From a methodological perspective, future research may further develop our framework for correcting for eye-tracker accuracy. The assumptions of our empirical method do not match the data perfectly and the method could be improved by taking into account the type of distributions of underlying dependent variables. Moreover, we know that several factors contribute to the validity of eye-trackers, e.g., data quality depends on the stimulus and the AOI size \citep{orquin2018a} and other artifacts such as sample population and recording location also matter \citep{nystroem2013a}. By extending our framework to include these other artifacts, it will be possible to make more precise estimates of effect sizes in meta-analysis and individual studies as well as more realistic power analyses.\\   

% \subsection{Limitations}
Some limitations of our findings have to be noted. All of the visual factors included a low number of studies which casts some doubt about the precision of the results. The low number of studies also means that the publication bias estimate is less reliable, thereby, adding to the uncertainty. This is unfortunate since recent findings suggest that meta-analytic results may considerably overestimate effect sizes compared to replication effect sizes, but that publication bias analysis largely reduces this difference \citep{kvarven2020}. An extenuating circumstance is that many of the included effect sizes were not central to or even hypothesized by the authors reporting them, which could imply that there was less selective reporting of these effects. One example is effect sizes for choice-gaze effect which many authors report as a by-product in descriptive statistics. Another challenge is that the studies included varied substantially e.g., high vs. low complexity stimuli or decision domain such as risky gambles vs. consumer choice. These differences may have introduced additional heterogeneity in the synthesized effect sizes, but at the same time, serve to increase the generalizability of the findings. \chg{r422c}{Finally, we wish to acknowledge that important process details may have been omitted from this analysis due to the selection of dependent variables. Specifically, two types of eye movement metrics are worth mentioning: fixation durations and scanpath metrics. Fixation durations can, for instance, reveal important aspects about decision processes such whether information is being processed in a more deliberate or intuitive manner \citep{horstmann2009}. However, not all studies report fixation durations and for our purpose the interpretation of fixation durations is complicated by the fact that it can reflect both visual and cognitive factors. Scanpath metrics such as the Search Index or Search Metric \citep{payne1976} are reported in even fewer studies, but can, for instance, be useful for distinguishing between compensatory and non-compensatory decision processes \citep{schoemann2019}.} \\ 

%%% 6. Finish with moving toward decision making in wild!
Our findings question several assumptions about how decision makers search for and gather information. The vast majority of existing theories and models assume, either implicitly or explicitly, that only cognitive factors matter. Most of the visual environment factors are ignored. While these models may work in a controlled laboratory environment, it is clear that they are not likely to generalize to more natural environments. \chg{call-to-action}{We hope our findings will inspire vision and decision scientists to collaborate on decision making models that integrate both visual and cognitive factors to improve our understanding of their interactions with the decision processes, and allow us to predict decision making in natural environments accurately. We also hope that decision researchers from the many disciplines included in this meta-analysis, such as psychology, neuroscience, economics, marketing, management, and political science, will become aware of the role of visual factors in their own context and examine visual factors with representative designs rather than experimentally eliminating them.}



%\subsection{Author contributions}

%JLO and ESL developed the study concept. ESL performed the literature search. JLO and ESL coded the studies. JLO and HS performed the analyses. JLO, ESL, and HS wrote the manuscript. 


% -----------------------------------------------------------
% Bibliography
% -----------------------------------------------------------

\bibliographystyle{apalike}
\bibliography{references.bib}
\clearpage


% -----------------------------------------------------------
% Appendix
% -----------------------------------------------------------

\appendix
% -----------------------------------------------------------
% Appendix
% -----------------------------------------------------------

\FloatBarrier
\section{}
\label{appendix}


\begin{figure}
\includegraphics{ET_accuracy_effectsize}
\centering
\caption{\textcolor{Red}{Accuracy of the eye-tracker affects the ability to reliably measure effect sizes in each study. Points denote the accuracy of an eye-tracker used in a study and the effect size measured with it. The line is based on the estimated intercept and slope from the best fitting mixed-effect model which was used to compute artifact multiplier, $a_a$. The multiplier was used to correct for a bias in estimated effect sizes due to differences in measurement accuracy of eye-trackers.}}
\label{fig:ET_accuracy_effectsize}
\end{figure}
\clearpage


% \caption{eye-tracker specifications table}
% \label{tab:eyetracker_specs}
% latex table generated in R 3.5.2 by xtable 1.8-4 package
% Mon Nov 30 09:54:56 2020
\begin{table}[ht]
\centering
\caption{Eye tracker specifications table, with accuracy and precision for each eye tracker as extracted from the manufacturer website, and computed artifact multiplier used for correcting for a bias in effect size estimates.} 
\label{tab:eyetracker_specifications}
\begin{tabular}{lccc}
  \hline
Eye tracker model & $a_a$ & Accuracy & Precision \\ 
  \hline
ASL6000 & 0.5523 & 1.00 & 0.50 \\ 
  Easygaze & 0.6866 & 0.70 & 0.35 \\ 
  Eye gaze 97 & 0.6794 & 0.72 & 0.50 \\ 
  Eye gaze tm & 0.8209 & 0.40 & 0.50 \\ 
  EyeLink 1000 & 0.7762 & 0.50 & 0.05 \\ 
  EyeLink 1000 (acc = .33) & 0.8523 & 0.33 & 0.05 \\ 
  EyeLink 1000 Plus (acc < .5) & 0.7762 & 0.50 & 0.05 \\ 
  EyeLink II & 0.7762 & 0.50 & 0.01 \\ 
  ISCAN & 0.5523 & 1.00 & 0.50 \\ 
  Nihon-Kohden EEG-1100 & 0.5523 & 1.00 & 0.50 \\ 
  SMI Glasses & 0.4583 & 1.21 & 0.19 \\ 
  SMI RED & 0.8209 & 0.40 & 0.03 \\ 
  SMI iview & 0.7762 & 0.50 & 0.01 \\ 
  SMI iview HED & 0.5523 & 1.00 & 0.50 \\ 
  SMI model unknown (acc < .5) & 0.7762 & 0.50 & 0.30 \\ 
  Tobii D10 & 0.7762 & 0.50 & 0.50 \\ 
  Tobii Glasses 1 & 0.3643 & 1.42 & 0.50 \\ 
  Tobii T120 & 0.8209 & 0.40 & 0.24 \\ 
  Tobii T1750 & 0.7762 & 0.50 & 0.25 \\ 
  Tobii T2150 & 0.7762 & 0.50 & 0.35 \\ 
  Tobii T60 & 0.7762 & 0.50 & 0.22 \\ 
  Tobii X1 & 0.7762 & 0.50 & 0.20 \\ 
  Tobii X2 & 0.8209 & 0.40 & 0.32 \\ 
  Tobii X60 & 0.7762 & 0.50 & 0.30 \\ 
  Tobii glasses 2 & 0.3643 & 1.42 & 0.34 \\ 
  Unknown & 0.6919 & 0.69 & 0.30 \\ 
   \hline 
 \multicolumn{4}{l}{\scriptsize{\textit{Note.} $a_a$ = artifact multiplier.}} 
\end{tabular}
\end{table}

\clearpage


\begin{figure}%[H]
\includegraphics{metric_correction}
\centering
\caption{\textcolor{Red}{Several eye movement metrics are used as dependent variable, but they are all highly correlated, suggesting they are all measuring the same underlying construct. Scatterplots show the relationship (A) between effect sizes expressed in fixation likelihood and fixation count, (B) between total dwell time and fixation likelihood, (C) between total dwell time and fixation count, (D) between total dwell time and dwell count. Lines in each plot represent the best-fitting linear regression line, and the shaded area 95\% confidence interval.}}
\label{fig:metric_correction}
\end{figure}
\clearpage


% \caption{Metric correction factor $a_m$ when correcting to either fixation count or fixation likelihood}
% \label{tab:metric_correction}
% latex table generated in R 3.6.3 by xtable 1.8-4 package
% Tue Jun  9 13:07:22 2020
\begin{table}[ht]
\centering
\caption{Metric correction factor $a_m$ when correcting to either fixation count or fixation likelihood} 
\label{tab:metric_correction}
\begin{tabular}{llr}
  \hline
Correcting from & Correcting to & $a_m$ \\ 
  \hline
Fixation count & Fixation likelihood & 0.97 \\ 
  Fixation likelihood & Fixation count & 1.03 \\ 
  Total fixation duration & Fixation likelihood & 0.81 \\ 
  Total fixation duration & Fixation count & 1.07 \\ 
  Dwell count & Fixation likelihood & 0.74 \\ 
  Dwell count & Fixation count & 0.98 \\ 
   \hline
\end{tabular}
\end{table}

\clearpage


% \caption{Moderator analysis results.}
% \label{tab:mod_results}
% latex table generated in R 3.6.3 by xtable 1.8-4 package
% Sat Jun 20 12:44:24 2020
\begin{table}[ht]
\centering
\caption{Moderator analysis results. The most important values are the corrected effect size estimate, $\rho$, and the associated heterogeneity, $I^2$. Results of trim and fill analysis are in the parentesis.} 
\label{tab:mod_results}
\begingroup\small
\begin{tabular}{lp{0.03\linewidth}p{0.05\linewidth}p{0.07\linewidth}p{0.07\linewidth}p{0.07\linewidth}p{0.07\linewidth}p{0.07\linewidth}p{0.07\linewidth}p{0.07\linewidth}}
  \hline
Group & $k$ & $N$ & $\rho$ & SE & $Z$ & $p$ & $\textrm{CI}_{95}$ LL & $\textrm{CI}_{95}$ UL & $I^2$ \\ 
  \hline
\textbf{Set size} &  &  &  &  &  &  &  &  &  \\ 
  \hspace{2mm}\textit{Alternative} & 6 (1) & 281 & 0.458 (0.489) & 0.089 & 5.172 & 0 & 0.285 (0.221) & 0.632 (0.688) & 8.23 \\ 
  \hspace{2mm}\textit{Attribute} & 4 (0) & 329 & 0.138 (0.209) & 0.119 & 1.155 & 0.248 & -0.096 (-0.132) & 0.371 (0.507) & 50.87 \\ 
  \textbf{Task instruction} &  &  &  &  &  &  &  &  &  \\ 
  \hspace{2mm}\textit{Alternative} & 12 (1) & 787 & 0.54 (0.565) & 0.072 & 7.497 & 0 & 0.399 (0.401) & 0.681 (0.694) & 0 \\ 
  \hspace{2mm}\textit{Attribute} & 14 (0) & 1203 & 0.383 (0.506) & 0.079 & 4.845 & 0 & 0.228 (0.302) & 0.538 (0.665) & 59.84 \\ 
  \textbf{Preferential viewing} &  &  &  &  &  &  &  &  &  \\ 
  \hspace{2mm}\textit{Alternative} & 7 (0) & 390 & 0.722 (0.882) & 0.121 & 5.993 & 0 & 0.486 (0.637) & 0.959 (0.965) & 62.32 \\ 
  \hspace{2mm}\textit{Attribute} & 14 (0) & 1624 & 0.426 (0.65) & 0.092 & 4.61 & 0 & 0.245 (0.387) & 0.607 (0.815) & 80.29 \\ 
   \hline 
 \multicolumn{10}{p{0.9\textwidth}}{\scriptsize{\textit{Note.} $k$ = number of studies (for trim and fill analysis number of imputed studies); $N$ = number of participants; $\rho$ = unattenuated effect size estimate, SE = standard error of estimate; $Z$ = Z statistic; $p$ = significance level; $\textrm{CI}_{95}$ LL = lower limit of the 95\% confidence interval; $\textrm{CI}_{95}$ UL = upper limit of the 95\% confidence interval, $I^2$ = within-group heterogeneity.}} 
\end{tabular}
\endgroup
\end{table}

\clearpage


% PET and PEESE result tables
%input{tables/PET-PEESE.tex}
%\clearpage


\begin{figure}%[H]
\includegraphics{forest_plots_altatt}
\centering
\singlespace
\caption{Effect sizes of the factors that were decomposed into alternative and attribute parts for moderator analyses. Forest plots show the unattenuated effect size correlations for each study in a group, as well as average effect across the group. Forest plot in (A) shows the effect sizes for set size -- alternative, in (B) for set size -- attribute, in (C) for task instructions -- alternative, in (D) for task instructions -- attribute, in (E) for preferential viewing -- alternative, and in (F) for preferential viewing -- attribute. Error bars represent the 95\% confidence interval around the mean.}
\label{fig:forest_plots_altatt}
\end{figure}
\clearpage


\begin{figure}[!h]
\includegraphics{funnel_plots}
\centering
\caption{Funnel plots for each factor that can be used as a qualitative check of a publication bias. Points are Fisher transformed correlation coefficients against their standard error. Asymmetric distributions of points can indicate the presence of publication bias since smaller studies (those with higher standard errors) have more variable effect sizes and are less likely to be published unless the effect is large. Funnel plot for (A) salience, (B) surface size, (C) left vs right position, (D) central position, (E) set size, (F) task instructions, (G) preferential viewing, and (H) choice-gaze effect.}
\label{fig:funnel_plots}
\end{figure}


\begin{figure}%[H]
\includegraphics{funnel_plots_altatt}
\centering
\singlespace
\caption{Funnel plots for factors that were decomposed into alternative and attribute parts for moderator analyses. Points are Fisher transformed correlation coefficients against their standard error. Asymmetric distributions of points can indicate the presence of publication bias since smaller studies (those with higher standard errors) have more variable effect sizes and are less likely to be published unless the effect is large. Funnel plot for (A) set size -- alternative, (B) set size -- attribute, (C) task instructions -- alternative, (D) task instructions -- attribute, (E) preferential viewing -- alternative, (F) preferential viewing -- attribute.}
\label{fig:funnel_plots_altatt}
\end{figure}
\clearpage


%\label{tab:overview}
% latex table generated in R 3.6.3 by xtable 1.8-4 package
% Sat Mar 27 13:38:54 2021
\begingroup\footnotesize
\begin{longtable}{p{5cm}lccclll}
\caption{Overview of individual effect sizes: IV = independent variable (LvR = Left vs. right position, Center = Center position, Sal = Salience, Pref = Preferential viewing, Choice = Choice-gaze effect, Task = Task instructions); $N$ = number of participants; $a_a$ = artifact multiplier; $r$ = attenuated effect size correlation expressed in the fixation count metric; Domain = research domain (Pref con = Preferential consumer choice, Pref non-con = Preferential non-consumer choice, Inf con = Inferential consumer choice, Inf non-con = Inferential non-consumer choice, Lotteries = Risky gambles); Alt/Att = Alternative or attribute manipulation.} \\ 
  \hline
Authors & IV & $N$ & $a_a$ & $r$ & Eye-tracker & Domain & Alt/Att \\ 
  \hline
\endhead
\hline
\multicolumn{8}{l}{\footnotesize Continued on next page}
\endfoot
\endlastfoot
 \hline
\cite{ares2014} & Pref & 71 & 0.776 & 0.320 & Tobii T60 & Pref con & Att \\ 
  \cite{ashby2015} & Pref & 27 & 0.821 & 0.069 & Eye gaze tm & Pref con & Att \\ 
  \cite{ashby2015} & Pref & 34 & 0.821 & 0.067 & Eye gaze tm & Pref con & Att \\ 
  \cite{ashby2015} & Pref & 81 & 0.821 & 0.104 & Eye gaze tm & Pref con & Att \\ 
  \cite{atalay2012a} & Center & 63 & 0.776 & 0.187 & Tobii T1750 & Pref con &  \\ 
  \cite{atalay2012a} & Center & 64 & 0.776 & 0.155 & Tobii T1750 & Pref con &  \\ 
  \cite{bagger2016} & Sal & 20 & 0.776 & 0.117 & EyeLink 1000 & Inf con & Att \\ 
  \cite{bagger2016} & Sal & 22 & 0.776 & 0.169 & EyeLink 1000 & Inf con & Att \\ 
  \cite{bagger2016} & Sal & 40 & 0.776 & 0.163 & EyeLink 1000 & Inf con & Att \\ 
  \cite{bagger2016} & Sal & 61 & 0.776 & 0.015 & EyeLink 1000 & Inf con & Att \\ 
  \cite{behe2014} & Pref & 330 & 0.776 & 0.094 & Tobii X1 & Pref con & Att \\ 
  \cite{behe2015} & Choice & 101 & 0.776 & 0.079 & Tobii X1 & Pref con &  \\ 
  \cite{behe2017} & Pref & 214 & 0.776 & 0.159 & Tobii X1 & Pref con & Att \\ 
  \cite{bialkova2011} & Task & 10 & 0.821 & 0.868 & SMI RED & Inf con & Att \\ 
  \cite{bialkova2014a} & Task & 80 & 0.821 & 0.347 & SMI RED & Inf con & Att \\ 
  \cite{bialkova2014a} & Task & 80 & 0.821 & 0.270 & SMI RED & Inf con & Att \\ 
  \cite{bialkova2020} & Task & 30 & 0.776 & 0.232 & SMI unknown & Pref con & Att \\ 
  \cite{bogomolova2020} & Sal & 200 & 0.821 & 0.138 & Tobii T120 & Pref con & Att \\ 
  \cite{brandstatter2014} & Choice & 80 & 0.776 & 0.775 & EyeLink II & Lotteries &  \\ 
  \cite{cavanagh2014} & Choice & 20 & 0.776 & 0.754 & EyeLink 1000 & Lotteries &  \\ 
  \cite{chandon2009a} & LvR & 348 & 0.552 & 0.019 & ISCAN & Pref con &  \\ 
  \cite{chandon2009a} & Center & 348 & 0.552 & 0.347 & ISCAN & Pref con &  \\ 
  \cite{chandon2009a} & Size & 348 & 0.552 & 0.346 & ISCAN & Pref con &  \\ 
  \cite{chandon2009a} & Task & 348 & 0.552 & 0.142 & ISCAN & Pref con & Alt \\ 
  \cite{du2014} & Pref & 72 & 0.821 & 0.228 & Tobii T120 & Pref con & Att \\ 
  \cite{fiedler2012} & Choice & 21 & 0.821 & 0.718 & Eye gaze tm & Lotteries &  \\ 
  \cite{fiedler2012} & Choice & 36 & 0.821 & 0.548 & Eye gaze tm & Lotteries &  \\ 
  \cite{folke2016} & Choice & 28 & 0.776 & 0.750 & EyeLink II & Pref con &  \\ 
  \cite{folke2016} & Choice & 24 & 0.776 & 0.808 & EyeLink 1000 & Pref con &  \\ 
  \cite{gidloef2017a} & Choice & 260 & 0.458 & 0.207 & SMI Glasses & Pref con &  \\ 
  \cite{gidloef2017a} & Center & 260 & 0.458 & 0.523 & SMI Glasses & Pref con &  \\ 
  \cite{gidloef2017a} & Pref & 260 & 0.458 & 0.095 & SMI Glasses & Pref con & Alt \\ 
  \cite{gidloef2017a} & Sal & 260 & 0.458 & 0.019 & SMI Glasses & Pref con &  \\ 
  \cite{gidloef2017a} & Size & 260 & 0.458 & 0.464 & SMI Glasses & Pref con &  \\ 
  \cite{gidlof2013} & Task & 40 & 0.552 & 0.040 & SMI iview HED & Pref con & Alt \\ 
  \cite{glaholt2009a} & Task & 16 & 0.776 & 0.000 & EyeLink 1000 & Inf non-con & Alt \\ 
  \cite{glaholt2009b} & Choice & 12 & 0.776 & 0.096 & EyeLink 1000 & Inf non-con &  \\ 
  \cite{glaholt2009b} & Choice & 12 & 0.776 & 0.153 & EyeLink 1000 & Pref non-con &  \\ 
  \cite{glaholt2009b} & Task & 12 & 0.776 & 0.153 & EyeLink 1000 & Inf non-con & Alt \\ 
  \cite{glaholt2009b} & Task & 12 & 0.776 & 0.455 & EyeLink 1000 & Inf non-con & Alt \\ 
  \cite{glaholt2009c} & Choice & 16 & 0.776 & 0.755 & EyeLink 1000 & Inf non-con &  \\ 
  \cite{glaholt2010} & LvR & 48 & 0.776 & 0.545 & EyeLink 1000 & Inf non-con &  \\ 
  \cite{glaholt2010} & Center & 48 & 0.776 & 0.645 & EyeLink 1000 & Inf non-con &  \\ 
  \cite{glaholt2012} & Choice & 24 & 0.776 & 0.646 & EyeLink 1000 & Inf non-con &  \\ 
  \cite{glaholt2012} & Task & 24 & 0.776 & 0.451 & EyeLink 1000 & Inf non-con & Alt \\ 
  \cite{graham2016} & Size & 155 & 0.776 & 0.106 & EyeLink 1000 & Pref con &  \\ 
  \cite{grebitus2015} & Set size & 130 & 0.776 & 0.019 & Tobii T60 & Pref con & Att \\ 
  \cite{guyader2017} & Task & 66 & 0.458 & 0.487 & SMI Glasses & Inf con & Alt \\ 
  \cite{hong2016a} & Set size & 75 & 0.821 & 0.440 & Tobii T120 & Pref con & Alt \\ 
  \cite{huang2011} & Task & 88 & 0.776 & 0.221 & EyeLink II & Pref non-con & Att \\ 
  \cite{hwang2017} & Task & 42 & 0.821 & 0.126 & Tobii X2 & Inf con & Att \\ 
  \cite{jenke2019} & Pref & 122 & 0.776 & 0.308 & Tobii T60 & Inf non-con & Att \\ 
  \cite{jenke2019} & Set size & 122 & 0.776 & 0.230 & Tobii T60 & Inf non-con & Att \\ 
  \cite{keller2014} & Task & 159 & 0.776 & 0.388 & SMI iview & Inf non-con & Att \\ 
  \cite{kim2012a} & Pref & 24 & 0.776 & 0.610 & EyeLink II & Lotteries & Alt \\ 
  \cite{krajbich2010a} & Choice & 39 & 0.776 & 0.926 & Tobii T2150 & Pref con &  \\ 
  \cite{kreplin2014a} & LvR & 19 & 0.552 & 0.270 & ASL6000 & Pref non-con &  \\ 
  \cite{kreplin2014a} & Center & 19 & 0.552 & 0.730 & ASL6000 & Pref non-con &  \\ 
  \cite{kwak2018} & LvR & 63 & 0.776 & 0.453 & Tobii T60 & Lotteries & Alt \\ 
  \cite{leboeuf2016} & Task & 54 & 0.776 & 0.652 & EyeLink 1000 & Inf con & Att \\ 
  \cite{lindner2014} & Choice & 30 & 0.776 & 0.453 & SMI iview & Inf non-con &  \\ 
  \cite{lindner2014} & Pref & 26 & 0.776 & 0.588 & SMI iview & Inf non-con & Alt \\ 
  \cite{lohse1997a} & Sal & 32 & 0.679 & 0.077 & Eye gaze 97 & Pref con &  \\ 
  \cite{lohse1997a} & Size & 32 & 0.679 & 0.174 & Eye gaze 97 & Pref con &  \\ 
  \cite{meissner2016a} & Center & 60 & 0.776 & 0.230 & EyeLink II & Pref con &  \\ 
  \cite{meissner2016a} & Pref & 60 & 0.776 & 0.798 & EyeLink II & Pref con & Alt \\ 
  \cite{meissner2016a} & Pref & 60 & 0.776 & 0.836 & EyeLink II & Pref con & Att \\ 
  \cite{meissner2016a} & Center & 35 & 0.821 & 0.120 & Tobii T120 & Pref con &  \\ 
  \cite{meissner2016a} & Pref & 35 & 0.821 & 0.775 & Tobii T120 & Pref con & Att \\ 
  \cite{meissner2016a} & Pref & 35 & 0.821 & 0.881 & Tobii T120 & Pref con & Alt \\ 
  \cite{meissner2016a} & Pref & 70 & 0.776 & 0.906 & Tobii T2150 & Pref con & Alt \\ 
  \cite{meissner2016a} & Pref & 70 & 0.776 & 0.930 & Tobii T2150 & Pref con & Att \\ 
  \cite{meissner2016a} & Center & 70 & 0.692 & 0.220 & Unknown & Pref con &  \\ 
  \cite{meissner2016b} & Set size & 40 & 0.821 & 0.420 & Tobii T120 & Pref con & Alt \\ 
  \cite{meyerding2018} & Pref & 73 & 0.364 & 0.301 & Tobii glasses 2 & Pref con & Att \\ 
  \cite{miller2015} & Pref & 358 & 0.776 & 0.382 & EyeLink 1000 & Pref con & Att \\ 
  \cite{mitsuda2014} & Choice & 48 & 0.776 & 0.842 & EyeLink II & Inf non-con &  \\ 
  \cite{neuhofer2020} & Size & 164 & 0.821 & 0.238 & Tobii X2 & Pref con & Att \\ 
  \cite{nittono2009} & Choice & 10 & 0.552 & 0.248 & Nihon-Kohden & Inf non-con &  \\ 
  \cite{nittono2009} & Task & 10 & 0.552 & -0.062 & Nihon-Kohden & Inf non-con & Alt \\ 
  \cite{orquin2013} & Pref & 68 & 0.776 & 0.195 & Tobii T2150 & Pref con & Att \\ 
  \cite{orquin2015a} & Sal & 150 & 0.776 & 0.067 & Tobii T60 & Pref con & Att \\ 
  \cite{orquin2015a} & Task & 100 & 0.776 & 0.052 & Tobii T60 & Pref con & Att \\ 
  \cite{orquin2019a} & Center & 91 & 0.776 & 0.267 & Tobii T2150 & Pref con & Att \\ 
  \cite{orquin2019a} & Sal & 91 & 0.776 & 0.088 & Tobii T2150 & Pref con & Att \\ 
  \cite{orquin2019a} & Set size & 91 & 0.776 & 0.078 & Tobii T2150 & Pref con & Att \\ 
  \cite{orquin2019a} & Size & 91 & 0.776 & 0.222 & Tobii T2150 & Pref con & Att \\ 
  \cite{orquin2019a} & Center & 76 & 0.776 & 0.068 & EyeLink 1000 & Inf con & Att \\ 
  \cite{orquin2019a} & Sal & 76 & 0.776 & 0.048 & EyeLink 1000 & Inf con & Att \\ 
  \cite{orquin2019a} & Set size & 76 & 0.776 & 0.020 & EyeLink 1000 & Inf con & Att \\ 
  \cite{orquin2019a} & Size & 76 & 0.776 & 0.230 & EyeLink 1000 & Inf con & Att \\ 
  \cite{orquin2019a} & Task & 52 & 0.776 & 0.083 & EyeLink 1000 & Inf con & Att \\ 
  \cite{orquin2020osfb} & Set size & 71 & 0.776 & 0.423 & Tobii T60 & Pref con & Alt \\ 
  \cite{orquin2020osfb} & Set size & 16 & 0.776 & 0.033 & Tobii T60 & Pref con & Alt \\ 
  \cite{orquin2020osfb} & Set size & 11 & 0.776 & 0.027 & Tobii T60 & Pref con & Alt \\ 
  \cite{orquin2020osfb} & Set size & 68 & 0.776 & 0.508 & Tobii T60 & Pref con & Alt \\ 
  \cite{paernamets2015a} & Task & 58 & 0.821 & 0.504 & SMI RED & Pref non-con & Alt \\ 
  \cite{paernamets2015a} & Task & 37 & 0.821 & 0.342 & SMI RED & Pref non-con & Alt \\ 
  \cite{peschel2019} & Sal & 127 & 0.776 & 0.004 & Tobii T60 & Pref con & Att \\ 
  \cite{peschel2019} & Size & 127 & 0.776 & 0.098 & Tobii T60 & Pref con & Att \\ 
  \cite{pieters1999} & Task & 54 & 0.692 & 0.051 & Unknown & Pref con & Att \\ 
  \cite{robertson2020} & LvR & 74 & 0.776 & 0.130 & EyeLink 1000* & Pref con & Att \\ 
  \cite{rubaltelli2012} & Task & 37 & 0.821 & 0.324 & Eye gaze tm & Lotteries & Att \\ 
  \cite{schoemann2019} & Task & 40 & 0.852 & 0.855 & EyeLink 1000** & Lotteries & Att \\ 
  \cite{schotter2010a} & Choice & 32 & 0.776 & 0.262 & EyeLink II & Inf non-con &  \\ 
  \cite{schotter2010a} & Choice & 32 & 0.776 & 0.290 & EyeLink II & Pref non-con &  \\ 
  \cite{schotter2012a} & Choice & 32 & 0.776 & 0.292 & EyeLink 1000 & Inf non-con &  \\ 
  \cite{spinks2016a} & Set size & 32 & 0.821 & 0.602 & Tobii T120 & Inf non-con & Att \\ 
  \cite{su2013} & Task & 49 & 0.776 & 0.454 & EyeLink II & Lotteries & Att \\ 
  \cite{turner2014} & Task & 89 & 0.776 & 0.534 & Tobii D10 & Pref con & Att \\ 
  \cite{vanderlaan2015} & Choice & 22 & 0.687 & 0.451 & Easygaze & Inf con &  \\ 
  \cite{vanderlaan2017} & Pref & 125 & 0.687 & 0.263 & Easygaze & Pref con & Alt \\ 
  \cite{vanherpen2011} & Task & 309 & 0.821 & 0.150 & SMI RED & Pref con & Att \\ 
  \cite{vanloo2015} & Pref & 81 & 0.821 & 0.257 & SMI RED & Pref con & Att \\ 
  \cite{vanloo2019} & Set size & 103 & 0.821 & 0.110 & Tobii X2 & Pref con & Att \\ 
  \cite{vanloo2019} & Pref & 103 & 0.821 & 0.132 & Tobii X2 & Pref con & Att \\ 
  \cite{wastlund2015} & Task & 98 & 0.364 & 0.247 & Tobii Glasses 1 & Inf con & Alt \\ 
  \cite{wastlund2015} & Task & 66 & 0.364 & 0.381 & Tobii Glasses 1 & Inf con & Alt \\ 
  \cite{wolfson2017} & Pref & 234 & 0.776 & 0.071 & EyeLink 1000 & Pref con & Att \\ 
  \cite{zuschke2020} & Sal & 172 & 0.776 & 0.262 & Tobii X60 & Pref con & Att \\ 
  \cite{zuschke2020} & Size & 172 & 0.776 & 0.303 & Tobii X60 & Pref con & Att \\ 
  \cite{zuschke2020} & Set size & 172 & 0.776 & 0.080 & Tobii X60 & Pref con & Att \\ 
  \hline
\label{tab:overviewtable}
\end{longtable}
\endgroup




\end{document}