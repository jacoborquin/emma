% -----------------------------------------------------------
% Introduction
% -----------------------------------------------------------
\section{Introduction}

% \section{Motivation and Problem}

Many of our decisions are made in environments where the relevant information must be acquired visually. In such visual environments choice alternatives can differ in their position, surface size, salience and many other visual properties. Consider, for instance, encountering a product with a surprising color on a supermarket shelf, or a restaurant menu where certain items take a prominent position and perhaps have an accompanying picture. Such visual properties have all been shown to influence our attention \citep{corbetta2002a,borji2012a,dehaene2003a,clarke2014a, rosenholtz2007a} and governments and companies are becoming more aware of how to use these visual properties to communicate effectively with citizens and consumers \citep{orquinwedel2020}. There is growing evidence showing that attention plays an important role in decision making \citep{gidloef2017a,krajbich2010a, stojic2020uncertainty, callaway2019a, gluth2018, gluth2020}, and can even causally affect choices \citep{ghaffari2018a, paernamets2015a, shimojo2003a}. However, the role of visual environment factors is almost completely absent from prominent decision theories. In most theories, cognitive factors such as goals in the decision task determine the relevance of objects and, either explicitly or implicitly, whether and when we look at them. Here, we ask whether decision research is building on correct assumptions about visual attention and the role of the visual environment, and provide an empirical assay of the relative importance of various visual and cognitive factors to guide theory development and real-world applications of visual factors.\\

% \section{Decision research (mostly) ignores bottom-up factors}
Most decision research considers attention to be determined by the decision process, that it is driven by the goal relevance of objects rather than their visual properties. In many prominent decision making models this assumption is implicit. Consider, for example, the prospect theory model of how probabilities and values of choice alternatives are integrated to arrive at a preferential choice \citep{tversky1979}. Alternatives are treated equally according to this model, and nothing in the model indicates that one piece of information should attract more attention than other. Prospect theory and related variants of expected utility theory focus on capturing the final choice, not the process of how people arrive at the choice. However, popular process-oriented decision models commit to similar assumptions about attention. Consider, for example, satisficing, elimination-by-aspect, or the lexicographic heuristics \citep{payne1988, simon1956a}. While these models all specify different information search processes, they make similar implicit assumptions about the nature of visual search and hence attention in decision making. The  models assume that information search is determined by a search rule inherent to the decision process, e.g. attend to alternatives one at a time until a satisfactory alternative is found \citep{stuttgen2012}, or attend to information cues in order of their predefined validity until a cue is found that identifies the best alternative \citep{krefeld-schwalb2019a}.\\ 

In recent sequential sampling models of decision making attention has had a more explicit role. Sequential sampling models assume that stochastic evidence for an alternative is accumulated over time and when the integrated evidence reaches a threshold a choice is made. This is a process-oriented model that aims to capture how people balance the value of accumulating more information with the cost of taking more time to reach a decision \citep{forstmann2016}. In two influential variants of these models attention plays an important role, by determining how evidence is sampled in favor of choice alternatives \citep{busemeyer1992} or by determining the weight assigned to the evidence \citep{krajbich2010a, thomas2019}. In these models, attention fluctuates randomly between choice alternatives or choice attributes until a choice is made. The implicit assumption being, that in the long run attention is uniformly distributed over alternatives and attributes. This is a stochastic equivalent to a maximizing decision rule such as the weighted additive which assumes that a decision maker attends equally to all information \cite{gloeckner2011a, payne1988}. In other words, even though attention exerts an influence on choice, this influence is random and neither controlled by goals nor the visual environment. Recently, sequential sampling models have been proposed in which attention is guided by the value of choice alternatives \citep{callaway2019a, gluth2018, gluth2020}. This assumption is supported by empirical findings demonstrating value based attentional capture, i.e. the effect that objects associated with rewards capture attention \citep{lepelley2015}. The models are reminiscent of an earlier idea by \cite{shimojo2003a} who proposed that decision makers attend preferentially to high value alternatives, which increases their value further, thus creating a feedback loop and increasing likelihood of gazing at the ultimately chosen alternative.\\ 

A few studies have proposed decision models where attention is not driven only by the goal relevance of alternatives, but also by their visual properties, focusing on salience, i.e. the visual conspicuousness of a stimulus relative to its surroundings. For example, \cite{towal2013a} showed that salience continuously influences the decision process by making some choice alternatives more likely to attract fixations, but it does not influence the drift rate, i.e. the speed of accumulating evidence, towards salient choice alternatives directly. \cite{chen2013} provided evidence that salience can determine the onset of drift towards a choice alternative, but not the drift rate itself. Finally, \cite{navalpakkam2010} showed that decision makers in a reward harvesting task made choices by combining value and salience, consistent with an ideal Bayesian observer. This work suggests that salience can also influence the decision process directly and not merely by biasing attention.\\ 

% \section{Why do we need bottom up factors in decision making models?}

The common assumption about cognitive factors being the only or main factor driving attention in decision making is inconsistent with a number of findings. \cite{vanderlans2008}, for instance, find that 2/3 of variance in attention is due to factors in the visual environment, unrelated to the decision task, and \cite{towal2013a} find that 1/3 of variance is due to visual factors. There are also several model free studies showing comparative effects of cognitive and visual factors on attention in decision making \citep{gidloef2017a, orquin2015a, orquin2019a}. Moreover, there is evidence that the visual environment influences choices by biasing visual attention. For instance, decision irrelevant visual factors have been shown to influence choices by changing the amount of gaze \citep{peschel2019, chandon2009a} or the order of gaze \citep{reeck2017a}. Even studies examining purely cognitive models of decision making often implicitly acknowledge the influence of visual factors by taking great effort to eliminate them by controlling the size, position, and salience of information \citep{brandstatter2014, gloeckner2011a, perkovic2018}.\\

Further evidence for the role of visual factors comes from vision science. The few studies that modelled the influence of the visual environment on attention in decision making focused exclusively on salience \citep{chen2013,navalpakkam2010, towal2013a}. This focus seems justified - a great deal of research in vision science has concentrated on salience, for a review see \cite{borji2012a}. The term salience refers to stimuli that differ from their surroundings in terms of visual conspicuity and it has been shown that observers are more likely to gaze at stimuli that are high in salience \cite{itti2000}. However, there has been much debate about the role of salience in guiding attention some arguing that it plays no role in, for instance, real-world behavior \citep{tatler2011a}. Besides salience, there are at least three other visual factors that are likely to guide attention in decision making \citep{orquin2013a, wedel2008}.\\

One factor is the relative surface size of stimuli, which refers to the proportion of the visual environment occupied by the stimulus \citep[for a review see][]{peschel2013a}. Increasing the surface size of choice alternatives has been shown to increase gaze by up to 25 \% \citep{chandon2009a}. Increments to surface size exhibit a diminishing marginal effect on eye movements \citep{lohse1997a}. A second factor is the position of stimuli which has been shown to influence eye movements and is sometimes corrected for in vision research models when estimating the influence of other variables of interest \citep{clarke2014a}. In a decision context alternatives are normally placed in different spatial locations, which means that position effects like left-to-right (reading) direction and centrality are likely to influence eye movements and choices \citep{atalay2012a, meissner2016a}. A third factor is the set size which in a decision context normally is operationalized as the number of alternatives or attributes. Increasing the set size generally slows reaction times to identify search targets \citep{wolfe2010} and may also increase the visual complexity by the addition of more and different visual stimuli. Visual complexity has been shown to increase the difficulty and amount of visual search \citep{rosenholtz2007a}. An important point about these visual factors is that all four are likely to vary in natural environments and have been shown to affect attention simultaneously \citep{orquin2019a}. While decision research often sees the visual environment as a nuisance factor and try to eliminate its influence on decision making \citep{brandstatter2014, gloeckner2011a, perkovic2018}, companies and governments often use the same factors to compete for the attention of consumers and citizens \citep{pieters2017, orquinwedel2020}.\\   

Despite these findings on the presumed importance of visual factors in attention and decision making, they have had only a small impact on theory development. While attention and its cognitive antecedents recently started playing a prominent role in decision theories \citep{callaway2019a, gluth2018, gluth2020, krajbich2010a, noguchi2018, thomas2019, usher2019}, the role of visual factors has been largely ignored. There are only a few studies that have proposed and tested models that incorporate the influence of the visual environment on attention in decision making \citep{chen2013, navalpakkam2010, towal2013a}. Moreover, these studies have focused exclusively on salience, despite the other visual factors that are likely to be relevant as well and their joint contribution. A systematic review that provides evidence on how important visual factors are individually, as well as relative to cognitive factors, would give a new impetus to theory development and real-world applications incorporating the role of the visual environment; or justify the lack of it. The increasing availability of eye-tracking equipment has paved the way for such a review. Eye-tracking provides a way to unobtrusively measure the influence of both visual and cognitive factors on attention in decision tasks. In the last two decades numerous model free eye-tracking studies appeared, situated in a decision context. These studies span many disciplines, from behavioural economics and consumer psychology to cognitive psychology, computational neuroscience and vision science, which potentially explains why such a review has not been done before.\\

% \section{Study approach} 

Here, we assess the importance of the visual environment in decision making by empirically examining the magnitude of effects of various visual factors on attention in decision making and comparing them with cognitive factors. We focus on four visual factors -- salience, position, surface size and set size -- and three cognitive factors -- task instruction effects, preferential viewing and choice bias. We collect effect sizes from studies on eye movements in decision making and meta-analyze them to get reliable effect estimates. To do so, we develop new methods to address methodological challenges of meta-analysing eye movement data. Our findings show that among the visual factors positioning a stimulus in the centre of the field of view has the largest effect, while salience has the smallest effect on attention. Relative to cognitive factors, visual factors have somewhat smaller effects on eye movements. However, since all visual factors can influence attention simultaneously, in cases with multiple factors \citep{gidloef2017a, orquin2019a}, these could jointly have a larger influence than cognitive factors. Overall, these results show that characteristics of the visual environment have reliable effects on eye movements in decision making and that the effects are present across various decision contexts and tasks. This suggests that future theories and models of decision making should integrate visual factors directly rather than see them as nuisance factors. Governments and companies can effectively guide decision makers' attention to information by positioning it centrally, by making it larger, by reducing the set size (competing information), and perhaps to some extent by making it more salient.  