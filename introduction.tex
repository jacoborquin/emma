% -----------------------------------------------------------
% Introduction
% -----------------------------------------------------------

\section{Introduction}

Many of our decisions are made in environments where the relevant information must be acquired visually. In such visual environments, choice alternatives can differ in their position, surface size, salience, and many other visual properties. Consider, for instance, encountering a product with a surprising color on a supermarket shelf, or a restaurant menu where certain items take a prominent position and perhaps have an accompanying picture. Such visual properties have all been shown to influence our attention \citep{corbetta2002a,borji2012a,dehaene2003a,clarke2014a, rosenholtz2007a}, and governments and companies are becoming more aware of how to use these visual properties to communicate effectively with citizens and consumers \citep{orquinwedel2020}. There is growing model-based evidence showing that attention is a critical component in decision processes \citep{krajbich2010a, stojic2020uncertainty, callaway2019a, gluth2018, gluth2020} as well as model-free evidence showing that attention can directly influence choices by determining which choice alternatives are in- or excluded from consideration \citep{chandon2009a, gidlof2013, gidloef2017a}.\footnote{There is also a growing number of studies using experimental manipulations to increase attention to a random alternative, which seems to have a small positive effect on choosing it \citep{armel2008, paernamets2015a, shimojo2003a}. This effect has been contested on methodological grounds \citep{Newell2018, ghaffari2018a}, but studies relying on different methods have been able to replicate the effect \citep{liu2020a, Fisher2020}.} However, the role of visual environment factors is almost completely absent from prominent decision theories as well as from most experimental work on decision making. In most decision theories, cognitive factors such as goals in the decision task determine the relevance of objects and, either explicitly or implicitly, whether and when we look at them. Here, we ask whether decision research is building on correct assumptions about visual attention and the role of the visual environment, and provide an empirical assay of the relative importance of various visual and cognitive factors to guide theory development and real-world applications of visual factors.

Most decision research does not concern itself with attention, but aims to predict choices and sometimes also choice processes. Many prominent decision making models implicitly assume attention to be determined by the decision process, that it is driven by the goal relevance of objects rather than their visual properties. Consider, for example, the prospect theory model of how probabilities and values of choice alternatives are integrated to arrive at a preferential choice \citep{tversky1979}. Alternatives are treated equally according to this model, and nothing in the model indicates that one piece of information should attract more attention than other. This implicitly assumes uniform attention across all pieces of information. Prospect theory and related variants of expected utility theory focus on capturing the final choice, not the process of how people arrive at the choice. However, popular process-oriented decision models commit to similar assumptions about attention. Consider, for example, satisficing, elimination-by-aspect, or the lexicographic heuristics \citep{payne1988, simon1956a}. While these models all specify different information search processes, they make similar implicit assumptions about the nature of visual search and hence attention in decision making. The models assume that information search is determined by a search rule inherent to the decision process, for example, attend to alternatives one at a time until a satisfactory alternative is found \citep{stuttgen2012}, or attend to information cues in order of their predefined validity until a cue is found that identifies the best alternative \citep{krefeld-schwalb2019a}. Overt attention is sometimes used to test such process assumptions \citep{gloeckner2011a}.

In recent years, attention has gained a more explicit role through sequential sampling models of decision making proposed in cognitive science and psychology. Sequential sampling models assume that stochastic evidence for an alternative is accumulated over time, and when the integrated evidence reaches a threshold, a choice is made. This is a process-oriented model that aims to capture how people balance the value of accumulating more information with the cost of taking more time to reach a decision \citep{forstmann2016}. In two influential variants of these models attention plays an important role, by determining how evidence is sampled in favor of choice alternatives \citep{busemeyer1992} or by determining the weight assigned to the evidence \citep{krajbich2010a, thomas2019}. In these models, attention fluctuates randomly between choice alternatives or choice attributes until a choice is made. The implicit assumption being, that in the long run attention is uniformly distributed over alternatives and attributes. This is a stochastic equivalent to a maximizing decision rule such as the weighted additive, which assumes that a decision maker attends equally to all information \citep{gloeckner2011a, payne1988}. In other words, even though attention exerts an influence on choice, this influence is random and neither controlled by goals nor the visual environment. Recently, sequential sampling models have been proposed in which attention is guided by the value of choice alternatives \citep{callaway2019a, gluth2018, gluth2020}. This assumption is supported by empirical findings demonstrating value-based attentional capture, that is, the effect that objects associated with rewards capture attention \citep{lepelley2015}. The models are reminiscent of an earlier idea by \cite{shimojo2003a} who proposed that decision makers attend preferentially to high-value alternatives, which increases their value further, thus creating a feedback loop and increasing likelihood of gazing at the ultimately chosen alternative.

The view of attention as being determined by the decision process has also been proposed in other disciplines concerned with decision making. In economics, the theory of rational inattention states that decision makers allocate limited attentional resources in a way that maximizes the expected utility of subsequent choices, taking into account information search costs \citep{sims2003}. Disciplines concerned with preference measurements, such as marketing, resource economics, hospitality research, and transportation research, often rely on random utility theory for understanding and predicting choices \citep{louviere2000}. It has become clear that decision makers often ignore product attributes of low subjective importance (termed attribute-non-attendance), and adaptations to random utility theory have been proposed based on inference or measurement of attribute-non-attendance \citep{vanloo2018}. In management and organization research, the behavioral theory of the firm proposes that limited attention by individuals in firms is allocated in a sequential process and that selection is driven by competing goals and the aspiration levels for each goal \citep{ocasio2011}. Attention is an important concept in all of these disciplines, and the prevailing view seems to be that attention is determined by cognitive factors such as utility maximization or goals.

The role of visual environment factors is almost completely absent from prominent decision theories. To be more precise, with visual factors we refer to any dimension of the choice stimulus which, unlike cognitive factors, can be sufficiently described in terms of its physical visual properties. Only a few studies have proposed decision models where attention is driven by such visual factors, alongside goal relevance of alternatives. These studies have focused on salience, that is, the visual conspicuousness of a stimulus relative to its surroundings. For example, \cite{towal2013a} showed that salience continuously influences the decision process by making some choice alternatives more likely to attract fixations, but it does not influence the drift rate, that is, the speed of accumulating evidence, towards salient choice alternatives directly. \cite{chen2013} provided evidence that salience can determine the onset of drift towards a choice alternative, but not the drift rate itself. Finally, \cite{navalpakkam2010} showed that decision makers in a reward harvesting task made choices by combining value and salience, consistent with an ideal Bayesian observer. This work suggests that salience can also influence the decision process directly and not merely by biasing attention.

The common assumption about cognitive factors being the only or main factor driving attention in decision making is inconsistent with a number of findings. \cite{vanderlans2008}, for instance, find that 2/3 of variance in attention is due to factors in the visual environment, unrelated to the decision task, and \cite{towal2013a} find that 1/3 of variance is due to visual factors. There are also several model-free studies showing comparative effects of cognitive and visual factors on attention in decision making \citep{gidloef2017a, orquin2015a, orquin2019a}. Moreover, there is evidence that the visual environment influences choices by biasing visual attention. For instance, decision irrelevant visual factors have been shown to influence choices by changing the amount of gaze \citep{peschel2019, chandon2009a} or the order of gaze \citep{reeck2017a}. Even studies examining purely cognitive models of decision making often implicitly acknowledge the influence of visual factors by taking great effort to eliminate them by controlling the size, position, and salience of information \citep{brandstatter2014, gloeckner2011a, perkovic2018}.

Further evidence for the role of visual factors comes from vision science. The few studies that modelled the influence of the visual environment on attention in decision making focused exclusively on salience \citep{chen2013,navalpakkam2010, towal2013a}. This focus seems justified - a great deal of research in vision science has concentrated on salience, for a review see \cite{borji2012a}. Most salience algorithms combine several low-level visual features by estimating pixel to pixel differences in color, contrast, edge density, and motion, and it has been shown that observers are more likely to gaze at locations that are high in salience \citep{itti2000}. However, there has been much debate about the role of salience in guiding attention, with some arguing that it plays no role in, for instance, real-world behavior \citep{tatler2011a}.

Besides salience, there are three other visual factors that have been studied in the context of decision making: surface size, position, and set size \citep{orquin2013a, wedel2008}. The surface size factor refers to the relative surface size of stimuli, that is, the proportion of the visual environment occupied by the stimulus \citep[for a review see][]{peschel2013a}. Increasing the surface size of choice alternatives has been shown to increase fixations by up to 25\% \citep{chandon2009a}. Increments to surface size exhibit a diminishing marginal effect on eye movements \citep{lohse1997a} and thus cannot be explained by chance probability of fixating an object. The position factor refers to the physical location of stimuli. It has been shown to influence eye movements and is sometimes corrected for in vision research models when estimating the influence of other variables of interest \citep{clarke2014a}. In a decision context, alternatives are normally placed in different spatial locations, which means that position effects like left-to-right (reading) direction and centrality are likely to influence eye movements and choices \citep{atalay2012a, meissner2016a}. The set size factor can be operationalized as the number of alternatives or attributes in a decision context. Increasing the set size generally slows reaction times to identify search targets \citep{wolfe2010}, and in decision tasks we expect that it will increase the total number of fixations in a trial, but decrease fixations to any given object as the proportion of fixated objects is likely to decrease \citep{spinks2016a}. The four visual factors can be operationalized independently of each other. For instance, it does not follow that increasing the surface size of an object will make it more salient nor does it change its position or the set size. All four factors are likely to vary in natural environments and have been shown to affect attention simultaneously \citep{gidloef2017a, orquin2019a}.

At the face of it, factors like salience, position, and surface size seem rather arbitrary in relation to decision making. However, it is important to realize that factors in natural visual environments are not random, but may be correlated with dimensions that are relevant to decision makers. For example, foraging monkeys rely on systematic visual differences between foliage and ripe fruit to detect food sources, as many plants advertise ripe fruits with colors such as red, yellow, or orange that may differ substantially in luminance from the leaves \citep{hiramatsu2008}. This example illustrates how low-level visual factors can be correlated with biologically important dimensions such as whether an object is edible or not. Other low-level visual factors have also been shown to predict important biological dimensions in natural visual scenes, such as object boundaries, distance, orientation, or velocity \citep{geisler2008}. In the context of value-based decision making there are also natural statistical relations between visual factors and cognitive dimensions relevant to decision makers. Visual salience, surface size, and position all predict product attribute class, $r_\textrm{salience} = .58$, $r_\textrm{size} = .73$, $r_\textrm{position} = .5$, because brands and logos are consistently more salient, larger, and more centrally positioned than health and sustainability attributes \citep{orquin2019a}. Surface size is correlated with product popularity, $r = .43$, with more popular product alternatives occupying a larger shelf area \citep{gidloef2017a}, and shelf position is correlated with prices, $r = .71$, with more expensive product alternatives placed closer to the top shelves \citep{valenzuela2013}. Given these high correlations between visual and cognitive factors in natural consumer environments, it is probably efficient for decision makers to allow visual factors to guide attention. Visual factors are also likely to be correlated with each other in natural consumer environments due to their correlation with cognitive factors e.g., popular products occupy larger shelf areas and are probably also more likely to be positioned on top shelves. In contrast, many paradigms in decision research seek to eliminate any correlation between visual and cognitive factors. This is not necessarily out of disregard for representative design \citep{dhami2004}, but because many tasks in decision research, such as risky gambles, intertemporal choice, or strategic choice, rarely have a natural visual representation. A consequence of experimentally eliminating correlations between visual and cognitive factors could be that decision makers cease to rely on visual factors for guiding attention \citep{bagger2016}. Furthermore, in these tasks, which are typically sparse in information compared to natural environments, decision makers are likely to attend all information \citep{fiedler2012}, and attention does therefore not mediate effects of the environment on choice. In fact, in such tasks it is often possible to predict choices equally well with or without attention \citep{glockner2012, glockner2014}. It is therefore not surprising that decision research often sees the visual environment as a nuisance factor and tries to eliminate its influence on decision making \citep{brandstatter2014, gloeckner2011a, perkovic2018}, while disciplines concerned with behavior in natural environments actively explore visual factors \citep{pieters2017, orquinwedel2020}.

Despite these findings on the presumed importance of visual factors in attention and decision making, they have had only a small impact on theory development. While attention and its cognitive antecedents recently started playing a prominent role in decision theories \citep{callaway2019a, gluth2018, gluth2020, krajbich2010a, noguchi2018, thomas2019, usher2019}, the role of visual factors has been largely ignored. There are only a few studies that have proposed and tested models that incorporate the influence of the visual environment on attention in decision making \citep{chen2013, navalpakkam2010, towal2013a}. Moreover, these studies have focused exclusively on salience, despite the other visual factors that are likely to be relevant as well and their joint contribution. A systematic review that provides evidence on how important visual factors are individually, as well as relative to cognitive factors, would give a new impetus to theory development and real-world applications incorporating the role of the visual environment; or justify the lack of it. The increasing availability of eye-tracking equipment has paved the way for such a review. Eye-tracking provides a way to unobtrusively measure the influence of both visual and cognitive factors on attention in decision tasks. In the last two decades numerous model-free eye-tracking studies appeared, situated in a decision context. These studies span many disciplines, from behavioral economics and consumer psychology to cognitive psychology, computational neuroscience, and vision science, which potentially explains why such a review has not been done before.

Here, we assess the importance of the visual environment in decision making by empirically examining the magnitude of effects of various visual factors on attention in decision making and comparing them with cognitive factors. We examine four visual factors: salience, position, surface size, and set size. To the best of our knowledge, these are the only visual factors examined in the context of decision making. We examine three cognitive factors: task instruction effects, preferential viewing, and choice-gaze effect. These three factors are fairly broad and therefore encompass most studies in decision making. We collect effect sizes from studies on eye movements in decision making and meta-analyze them to get reliable effect estimates. To do so, we develop new methods to address methodological challenges of meta-analyzing eye movement data. Based on previous studies comparing visual and cognitive factors \citep{vanderlans2008, towal2013a}, we expect that visual factors have reliable effects that are at least comparable in magnitude to those of cognitive factors. If this is true, understanding choices in the real world would call for a substantial change in theories and models of decision making, it would require integrating visual factors directly rather than to see them as nuisance factors. Governments and companies will also be able to effectively guide decision makers' attention to important information by relying on visual factors.  
